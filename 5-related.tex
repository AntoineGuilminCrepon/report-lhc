%!TEX root = main.tex
\section{Related work}

There is a large body of research regarding relational logics. We compare here our work to closely related research.

\emph{Relational Hoare Logic and derivatives.} The original Relational Hoare Logic (RHL) from \citet{Benton04} provides a framework for judgments on pairs of programs, supporting lockstep proofs. This logic, and its extension to Separation logic by \citet{Yang07} have already been used to prove a wide range of properties.\citet{BartheCK11} introduces product programs as a way to handle 2-properties, by reducing them to standard Hoare logic statements. Futher work in \citet{BartheKOB13, BartheGHS17} extends this concept to probabilistic properties, like security of cryptographic code. Cartesian Hoare Logic (CHL), presented in \citet{SousaD16}, generalizes RHL to $k$-safety property, with fixed $k$. 
