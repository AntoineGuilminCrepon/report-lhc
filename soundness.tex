%TEX root = main.tex
\section*{Soundness proofs}

\begin{itemize}
    \item \textsc{wp-triv}:

            If $\m{t}$ is deterministic and has a terminating trace, then any terminating trace gives the same output, given the same input. Thus $\mathbf{wp}_{\exists}$ and $\mathbf{wp}_{\forall}$ have the same semantic value.
%          \begin{eqnarray*}
 %             \V |- \WPU{\m{t}}{\WPE {\m{t}} {\True}} &:=& \A {s_0}\ v_1\ s_1. \E v_2\ s_2. \bigstep t {s_0} {v_1} {s_1} \Rightarrow \bigstep t {s_1} {v_2} {s_2} \\
 %             &\Leftarrow& \A s\ v\ s'. \bigstep t s v {s'} \Rightarrow \bigstep t s v {s'}
 %         \end{eqnarray*}

    \item \textsc{wp$_{\exists}$-cons}:

          Suppose $\V \A v. Q(v) |- Q'(v)$.

            $\WPE {\m{t}} Q$ means there exists an return value $v$ and a output store $s'$ s.t $\bigstep t s v {s'}$ and $Q(v)(s')$. By the first assumption, one gets $Q'(v)(s')$, thus $\WPE {\m{t}} {Q'}$.

    \item \textsc{wp$_{\exists}$-exists}:

          Given that $\m{t}$ does not depend on $x$,
          \begin{eqnarray*}
              && \E x.\WPE{\m{t}}{Q(x)} \lequiv \WPE{\m{t}}{\E x.Q(x)} \\
              &:=& \A s. (\E x\ v\ s'. \bigstep t s v {s'} \land Q(x)(v)(s') \iff \E v\ s'. \bigstep t s v {s'} \land (\E x. Q(x)(v)(s')))
          \end{eqnarray*}

    \item \textsc{wp$_{\exists}$-frame}:

          Suppose $\V \Gamma |- \WPE{\m{t}}{Q}$ and $\pvar(P) \cap \mods(t) = \emptyset$. The first assumption implies a terminating trace for $\m{t}$ verifying $Q$.

          Given that no variable of $P$ is modified by $\m{t}$, for every execution of the term, $P$ will equally hold at both ends of the trace.

          Thus $\V \Gamma,P |- \WPE{\m{t}}{\lambda r. P \land Q(r)}$.
    
    \item \textsc{wp$_{\exists}$-impl-r}:

    Similar to \textsc{wp$_{\exists}$-frame}.

    \item \textsc{wp$_{\exists}$-subst}:

          Suppose $\p{x}(i) = v \land \WPE{([i: t\subst{\p{x}->v}]\m.\m{t}')}{Q}$. We have therefore a trace of $[i: t\subst{\p{x}->v}]\m.\m{t}'$ that satisfies $Q$ given that we also suppose that $\p{x}(i)=v$ and that $\p{x}$ is not modified by $t$, we can simulate the trace given in the new hyperterm $[i: t]\m.\m{t}'$, by taking the value of $\p{x}$ from the store (where it will be $v$ for the entire execution), instead of having it directly in the term.

    \item \textsc{wp$_{\exists}$-idx}:
          
          The proof relies on the fact that the existence of some terminating trace for a certain hyperterm is unchanged by a bijective reindexing of said hyperterm.

    \item \textsc{wp$_{\exists}$-seq$_I$}:
          
          Suppose that $\WPE {\m[i: (t_i\code{;}\ t'_i) | i \in I]} {Q}$. That gives us a trace of $t_i; t_i'$ satisfying $Q$. We can cut this trace in two part, the one executing $t_i$, the other executing $t_i'$. We can then deduce an intermediary store that registers the state of the program between the two terms, allowing use to write $\WPE {\m[i: t_i | i \in I]}{\WPE {\m*[i: \smash{t'_i} | i\in I]} {Q}}$. The reciproqual is straight-forward.

    \item \textsc{wp$_{\exists}$-nest}:
    
          Suppose $\WPE{(\m{t}_1 \cdot \m{t}_2)}{Q}$. It says that for all input states, there exists an output state satisfying $Q$.

          Given that there are no side effects between the terms of an hyperterm, one can "divide" the trace as two separate traces in $\m{t}_1$ and $\m{t}_2$. Thus, forall input state, there exists a state that coincides with the output state given by the assumption for the domain of $\m{t}_1$, and with the input state for the domain of $\m{t}_2$ (it is here of prime importance that $\m{t}_1$ and $\m{t}_2$ are disjoint).

          Therefore, $\WPE{\m{t}_1}{\WPE{\m{t}_2}{Q}}$
    \item \textsc{wp$_{\exists}$-conj}:

          Suppose $\WPE{\m{t}_1}{Q_1} \land \WPE{\m{t}_2}{Q_2}$. It gives a trace for $\m{t}_1$ satisfying $Q_1$, and a trace for $\m{t}_2$ satisfying $Q_2$.
          
          To prove the result, one should construct a trace of $\m{t}_1 + \m{t}_2$ that satisfies $Q_1 \land Q_2$. To do so, combining the traces given by the assumptions works, apart from the intersections of the domains of the two hyperterms. For this we need the added assumption of determinism, that guarantees the uniqueness of the trace. Therefore we get the result.

    \item \textsc{wp$_{\forall}$-wp$_{\exists}$-swap}:

          Suppose $\WPE {\m{t}_2} {\WPU {\m{t}_1} {Q}}$. It gives us a trace of $\m{t}_2$ such that every trace of $\m{t}_1$ executes afterwards validates $Q$.

          For terms outside of the intersection of $\m{t}_1$ and $\m{t}_2$, the two traces can be executed in any order. For the terms in the intersection, the assumption of determinism allows for switching $\mathbf{wp}_{\forall}$ and $\mathbf{wp}_{\exists}$ locally, to give the final result.
\end{itemize}