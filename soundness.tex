%TEX root = main.tex
\section*{Soundness proofs}

\begin{itemize}
    \item \textsc{wp-triv}:
          \begin{eqnarray*}
              \V |- \WPU{\m{t}}{\WPE {\m{t}} {\True}} &:=& \A s\ v_1\ s'_1. \E v_2\ s'_2. \bigstep t s {v_1} {s_1'} \Rightarrow \bigstep t s {v_2} {s_2'} \\
              &\Leftarrow& \A s\ v\ s'. \bigstep t s v {s'} \Rightarrow \bigstep t s v {s'}
          \end{eqnarray*}

    \item \textsc{wp$_{\exists}$-cons}:

          Suppose $\V \A v. Q(v) |- Q'(v)$.

            $\WPE {\m{t}} Q$ means there exists an return value $v$ and a output store $s'$ s.t $\bigstep t s v {s'}$ and $Q(v)(s')$. By the first assumption, one gets $Q'(v)(s')$, thus $\WPE {\m{t}} {Q'}$.

    \item \textsc{wp$_{\exists}$-exists}:

          Given that $\m{t}$ does not depend on $x$,
          \begin{eqnarray*}
              && \E x.\WPE{\m{t}}{Q(x)} \lequiv \WPE{\m{t}}{\E x.Q(x)} \\
              &:=& \A s. (\E x\ v\ s'. \bigstep t s v {s'} \land Q(x)(v)(s') \iff \E v\ s'. \bigstep t s v {s'} \land (\E x. Q(x)(v)(s')))
          \end{eqnarray*}

    \item \textsc{wp$_{\exists}$-frame}:

          Suppose $\V \Gamma |- \WPE{\m{t}}{Q}$ and $\pvar(P) \cap \mods(t) = \emptyset$. The first assumption implies a terminating trace for $\m{t}$ verifying $Q$.

          Given that no variable of $P$ is modified by $\m{t}$, for every execution of the term, $P$ will equally hold at both ends of the trace.

          Thus $\V \Gamma,P |- \WPE{\m{t}}{\lambda r. P \land Q(r)}$.
    
    \item \textsc{wp$_{\exists}$-impl-r}:

    Similar to \textsc{wp$_{\exists}$-frame}.

    \item \textsc{wp$_{\exists}$-idx}:
          
          The proof relies on the fact that the existence of some terminating trace for a certain hyperterm is unchanged by a bijective reindexing of said hyperterm.

    \item \textsc{wp$_{\exists}$-nest}:
    
          Suppose $\WPE{(\m{t}_1 \cdot \m{t}_2)}{Q}$. It says that for all input states, there exists an output state satisfying $Q$.

          Given that there are no side effects between the terms of an hyperterm, one can "divide" the trace as two separate traces in $\m{t}_1$ and $\m{t}_2$. Thus, forall input state, there exists a state that coincides with the output state given by the assumption for the domain of $\m{t}_1$, and with the input state for the domain of $\m{t}_2$ (it is here of prime importance that $\m{t}_1$ and $\m{t}_2$ are disjoint).

          Therefore, $\WPE{\m{t}_1}{\WPE{\m{t}_2}{Q}}$
    \item \textsc{wp$_{\exists}$-conj}:

          Suppose $\WPE{\m{t}_1}{Q_1} \land \WPE{\m{t}_2}{Q_2}$. It gives a trace for $\m{t}_1$ satisfying $Q_1$, and a trace for $\m{t}_2$ satisfying $Q_2$.
          
          To prove the result, one should construct a trace of $\m{t}_1 + \m{t}_2$ that satisfies $Q_1 \land Q_2$. To do so, combining the traces given by the assumptions works, apart from the intersections of the domains of the two hyperterms.
\end{itemize}