%TEX root = main.tex
\section{Description of the extension}

The main point of the extension is to add an object to the language complementary to the $\mathbf{wp}_{\forall}$ from the original model.

This object is defined as such :

$\WPE{\m{t}}{Q} ::= \lambda s. \E s' v. \bigstep t s v {s'} \land Q(v)$

This object behaves in the same way as $\mathbf{wp}_{\forall}$ for most of the rules previously described. Those rules are written down for completeness.

\begin{mathfig}{\small}
    \begin{proofrules}
        %!TEX root = ../main.tex
\infer*[lab=wp$_{\exists}$-cons]{
  \A \m{v}.
  \V Q(\m{v}) |- Q'(\m{v})
}{
  \V \WPE{\m{t}}{Q} |- \WPE{\m{t}}{Q'}
}

        %!TEX root = ../main.tex
\infer*[lab=wp$_{\exists}$-exists]{}
{
  \E x.\WPE{\m{t}}{Q(x)} \lequiv \WPE{\m{t}}{\E x.Q(x)}
}
%   \exists x\st\ %
%   \V \Gamma |- \WPE{\m{t}}{Q(x)}
% }{
%   \V \Gamma |- \WPE{\m{t}}{\A x.Q(x)}
% }

        %!TEX root = ../main.tex
\infer*[lab=wp$_{\exists}$-frame]{
  \V \Gamma |- \WPE{\m{t}}{Q}
  \\
  \pvar(P) \inters \mods(\m{t}) = \emptyset
}{
  \V \Gamma,P |- \WPE {\m{t}} {\ret. P \land Q(\ret)}
}

        %!TEX root = ../main.tex
\infer*[lab=wp$_{\exists}$-impl-r]{
  \pvar(P) \inters \mods(\m{t}) = \emptyset
}{
  \V \proj(\m{t}),
  P \implies \WPE{\m{t}}{Q}
  |-
  \WPE{\m{t}}{\ret. P \implies Q(\ret)}
}

        %!TEX root = ../main.tex
\infer*[lab=wp$_{\exists}$-subst]{
  \p{x} \notin \mods(t)
}{
  \V
  \p{x}(i) = v
  \land
  \WPE{\bigl(\map[\big]{i: t\subst{\p{x}->v}}\m.\m{t}'\bigr)}{Q}
  |-
  \WPE{(\m[i: t]\m.\m{t}')}{Q}
}


        %!TEX root = ../main.tex
\infer*[lab=wp$_{\exists}$-idx]{
  \pi\ \mathrm{bijective}
}{
  \V (\WPE{\m{t}}{Q})\isub*{\pi} |- \WPE{\m{t}\isub*{\pi}}{Q\isub*{\pi}}
}

        %!TEX root = ../main.tex
\infer*[lab=wp$_{\exists}$-seq$_I$]{}{
  \WPE {\m[i: t_i | i \in I]}[\big]{
    \WPE {\m*[i: \smash{t'_i} | i\in I]} {Q}
  }
  \lequiv
  \WPE {\m[i: (t_i\code{;}\ t'_i) | i \in I]} {Q}
}

        %!TEX root = ../main.tex
\infer*[lab=wp$_{\exists}$-assign$_I$]{
  \exists i\in I\st
  (\p{x}_i,i) \not\in \pvar(Q)
}{
  \V
  \WPE {\m[i: e_i | i\in I]} {Q}
  |-
  \WPE {\m[i: \p{x}_i \code{:=}\, e_i | i \in I]}
  {\ret.Q(\ret) \land \LAnd_{i\in I} \ret[i] = \p{x}_i(i)}
}

        %!TEX root = ../main.tex
\infer*[lab=wp$_{\exists}$-if$_I$]{}{
  \WPE {\m[i: g_i | i\in I]}*{
    \fun\m{b}.
    \WPE {
      \begin{pmatrix*}[l]
        {\m[i: t_i | i\in I, \m{b}(i) \neq 0]}
        {\m.}\\
        {\m[i: t'_i | i\in I, \m{b}(i) = 0]}
      \end{pmatrix*}
    }[\big]{Q}
  }
  \lequiv
  \WPE {\m[i: \code{if}\ g_i\ \code{then}\ t_i\ \code{else}\ t'_i | i \in I]}
  {Q}
}

        %!TEX root = ../main.tex
\infer*[lab=wp$_{\exists}$-while$_I$]{
  \V
  P
  |-
  \WPE {\m[i: g_i | i\in I]}[\big]{
    \fun\m{b}.
    (\m{b} =_I 0 \land R)
    \lor
    (\m{b} \ne_I 0 \land \WPE {\m[i: t_i | i \in I]} {P})
  }
}{
  \V
  P
  |-
  \WPE {\m[i: \code{while}\ g_i\ \code{do}\ t_i | i \in I]} {R}
}

        %!TEX root = ../main.tex
\infer*[lab=wp$_{\exists}$-nest]{%
  % \supp(\m{t}_1) \inters \supp(\m{t}_2) = \emptyset
  % implied by well definedness
}{
  % \WPE{\m{t}_1}{\WPE{\m{t}_2}{Q}} % simpler but discards ret vals of t1
  \WPE{\m{t}_1}{\fun\m{v}.
    \WPE{\m{t}_2}{\fun\m{w}.
      Q(\m{v}\m.\m{w})}}
  \lequiv
  \WPE{(\m{t}_1 \m. \m{t}_2)}{Q}
}

        %!TEX root = ../main.tex
\infer*[lab=wp$_{\exists}$-conj]{
  % \idx(Q_1) \subs \supp(\m{t}_1)
  \idx(Q_1) \inters \supp(\m{t}_2) \subs \supp(\m{t}_1)
  \\
  % \idx(Q_2) \subs \supp(\m{t}_2)
  \idx(Q_2) \inters \supp(\m{t}_1) \subs \supp(\m{t}_2)
}{
  \V
  \WPE{\m{t}_1}{Q_1}
  \land
  \WPE{\m{t}_2}{Q_2}
  |-
  \WPE{(\m{t}_1 \m\cdot \m{t}_2)}{Q_1 \land Q_2}
}

        %!TEX root = ../main.tex
\infer*[lab=wp$_{\exists}$-idx-pass]{
  i,j \notin \supp(\m{t})
}{
  \V (\WPE{\m{t}}{Q})\isub{j->i} |- \WPE{\m{t}}{Q\isub{j->i}}
}


        %!TEX root = ../main.tex
\infer*[lab=wp$_{\exists}$-idx-swap]{
  i \notin \idx(Q)
  % \\
  % i \notin \supp(\m{t}') % implied by well definedness
}{
  \V
  \bigl(\WPE{(\m[j: t]\m.\m{t}')}{Q}\bigr)\isub{j->i}
  |-
  \WPE{(\m[i: t]\m.\m{t}')}{Q\isub{j->i}}
}


        %!TEX root = ../main.tex
\infer*[lab=wp$_{\exists}$-idx-merge]{}{
  \V
  \bigl(\WPE{(\m[i: t, j: t]\m.\m{t}')}{Q}\bigr)\isub{j->i}
  |-
  \WPE{(\m[i: t]\m.\m{t}')}{Q\isub{j->i}}
}


        %!TEX root = ../main.tex
\infer*[lab=wp$_{\exists}$-idx-post]{
  \V \Gamma |- \WPE{\m{t}}{Q}
  \\
  j \notin \supp(\m{t}) \union \idx(\Gamma)
}{
  \V \Gamma |- \WPE{\m{t}}{Q\isub{j->i}}
}

    \end{proofrules}
    \caption{Rules for $\mathbf{wp}_{\exists}$ mirrored on $\mathbf{wp}_{\forall}$}
\end{mathfig}

An interesting point is that predicates that couldn't be defined in the original theory using the base langugage, and thus needed to be explicitly defined to be used, can now be defined using this new $\mathbf{wp}_{\exists}$ object.

\begin{eqnarray*}
    \proj(t) &:=& \WPE{\m{t}}{\True} \\
    t_1 \semleq t_2 &\lequiv& \WPU{[1: t_1]}{\WPE{[2: t_2]}{v(1) = v(2) \land s(1) = s(2)}}
\end{eqnarray*}

\begin{mathfig}{\small}
    \begin{proofrules}
        %!TEX root = ../main.tex
\infer*[lab=wp$_{\forall}$-triv]{}{
  \V |- \WPU{\m{t}}{\WPE {\m{t}} {\True}}
}
        
        %!TEX root = ../main.tex
\infer*[lab=wp$_{\forall}$-wp$_{\exists}$-swap]
{
    \A i. \WPU {[1: \m{t}_2(i)]} {\WPE {[2:\m{t}_1(i)]} {Q}}
}{
  \V \WPE {\m{t}_2} {\WPU {\m{t}_1} {Q}}|- \WPU{\m{t}_1}{\WPE {\m{t}_2} {Q}}
}
    \end{proofrules}
    \caption{Rules combining $\mathbf{wp}_{\forall}$ and $\mathbf{wp}_{\exists}$ rules}
\end{mathfig}