%TEX root = main.tex
\section{Existential extension of LHC}
\label{sec:extension}

\subsection{New operator}

The main point of the extension is to add a modality to the language, dual to the $\wpsymb_{\forall}$ from the original model. It is defined as such :

\begin{definition}[Weakest existential precondition]
    $\WPE{\m{t}}{Q} := \fun \m{s}. \E \m{s'} \m{v}. \bigstep {\m{t}}{\m{s}} {\m{v}} {\m{s'}} \land Q(\m{v})$
\end{definition}

One can easily appreciate the similarity with the original $\wpsymb_{\forall}$ definition. Intuitively, the judgment $\WPE{\m{t}}{Q}$ states that there exists a trace of $\m{t}$ satisfying $Q$. An important point is that any judgment of this form implies \emph{at least} local projectability of $\m{t}$, \ie if we chose an input state satisfying the constraint of the potential context of the judgment, $\m{t}$ must have a terminating trace beginning in this state.

When finding the rules defining the behavior of $\wpsymb_{\exists}$, it seems that most rules are direct mirrors of the $\wpsymb_{\forall}$ rules (simply replacing one predicate with the other). These "mirrored" rules are written in \cref{sec:mirrored-rules}, for completeness.

\begin{mathfig}{\small}
    \begin{proofrules}
        %!TEX root = ../main.tex
\infer*[lab=wp$_{\exists}$-exists]{}
{
  \E x.\WPE{\m{t}}{Q(x)} \lequiv \WPE{\m{t}}{\E x.Q(x)}
}
%   \exists x\st\ %
%   \V \Gamma |- \WPE{\m{t}}{Q(x)}
% }{
%   \V \Gamma |- \WPE{\m{t}}{\A x.Q(x)}
% }
        \label{rule:wpE-exists}

        %!TEX root = ../main.tex
\infer*[lab=wp$_{\exists}$-impl-l]{
  \pvar(P) \inters \mods(\m{t}) = \emptyset
}{
  \WPE{\m{t}}{\ret. Q(\ret) \implies P}
  \lequiv
  \WPE{\m{t}}{Q} \implies P
}
        \label{rule:wpE-impl-l}

        %!TEX root = ../main.tex
\infer*[lab=wp$_{\exists}$-while$_I$]{
  \V
  {\A \alpha. P(\alpha)}
  |-
  \WPE {\m[i: g_i | i\in I]}[\big]{
    \fun\m{b}.
    (\m{b} =_I 0 \land R)
    \lor
    (\m{b} \ne_I 0 \land \WPE {\m[i: t_i | i \in I]} {\E \alpha'. {P(\alpha') \land \alpha' < \alpha}})
  }
}{
  \V
  {P (\alpha_0)}
  |-
  \WPE {\m[i: \code{while}\ g_i\ \code{do}\ t_i | i \in I]} {R}
}
        \label{rule:wpE-while}

        %!TEX root = ../main.tex
\infer*[lab=wp$_{\exists}$-conj]{
  % \idx(Q_1) \subs \supp(\m{t}_1)
  \idx(Q_1) \inters \supp(\m{t}_2) \subs \supp(\m{t}_1)
  \\
  % \idx(Q_2) \subs \supp(\m{t}_2)
  \idx(Q_2) \inters \supp(\m{t}_1) \subs \supp(\m{t}_2)
}{
  \V
  \WPE{\m{t}_1}{Q_1}
  \land
  \WPE{\m{t}_2}{Q_2}
  |-
  \WPE{(\m{t}_1 \m+ \m{t}_2)}{Q_1 \land Q_2}
}

        \label{rule:wpE-conj}
    \end{proofrules}
    \caption{Rules specific to $\wpsymb_{\exists}$}
    \label{fig:wpE-rules}
\end{mathfig}

\Cref{fig:wpE-rules} gives out the rules applying to $\wpsymb_{\exists}$ that are substantially different from their $\wpsymb_{\forall}$ couterparts. \Cref{rule:wpE-exists,rule:wpE-impl-l} show that $\wpsymb_{\exists}$ commutes with $\exists$ and $(\implies P)$ (note the duality with \cref{rule:wpU-all,rule:wpU-impl-r}). Because any $\wpsymb_{\exists}$ judgment implies the projectability of the hyperterm, the "invariant" $P$ in \cref{rule:wpE-while} needs to be parametrized by a decreasing loop index $\alpha$. This loop index needs of course to be in a well-founded order (most of the time, natural numbers should suffice, but the possibility to call on transfinite ordinals reinforces the rule). \Cref{rule:wpE-conj} is slightly weaker than its $\wpsymb_{\forall}$ sibling, because the two hyperterms now requires to be disjoint. The reason is that the two traces given by the premisse $\WPE{\m{t}_1}{Q_1} \land \WPE{\m{t}_2}{Q_2}$ do not require to coincide on $\m{t}_1\cap\m{t}_2$

In the original logic, certain modalities had to be defined outside the logic, \emph{ad hoc}. It was the case for the projectability predicate and the refinement predicate. With the new addition of $\wpsymb_{\exists}$, one can now rewrite those modalities inside the logic.

The easiest one is $\proj(\m{t})$. If we unfold the definition, we realise that it is exactly the definition of $\WPE{\m{t}}{\True}$. Furthermore, one can state a refinement judgment $t_1 \semleq t_2$ as $\pv{x}(\I 1) = \pv{x}(\I 2) \implies \WPU {\m[\I 1: t_1]} {\WPE {\m[\I 2: t_2]} {\pv{x}(\I 1) = \pv{x}(\I 2)}}$, under the assumption that $\pvar(t_1)\cup\pvar(t_2)\subseteq\pv{x}$. The validity of this refinement will be further discussed in \cref{sec:applications}.

\begin{mathfig}{\small}
    \begin{proofrules}
        %!TEX root = ../main.tex
\infer*[lab=overapprox-refine]{}{
  \V
  \pv{x}(\I 1) = \pv{x}(\I 2) \implies \WPU{[1: t_1]}{\WPE{[2: t_2]}{\pv{x}(\I 1) = \pv{x}(\I 2)}},
  \WPU {(\m*[i: t_2] \m. \m{t})} {Q}
  |-
  \WPU {(\m[i: t_1] \m. \m{t})} {Q}
}
        \label{rule:overapprox-refine}

        %!TEX root = ../main.tex
\infer*[lab=underapprox-refine]{}{
  \V
  \pv{x}(\I 1) = \pv{x}(\I 2) \implies \WPU{[1: t_1]}{\WPE{[2: t_2]}{\pv{x}(\I 1) = \pv{x}(\I 2)}},
  \WPE {(\m*[i: t_1] \m. \m{t})} {Q}
  |-
  \WPE {(\m[i: t_2] \m. \m{t})} {Q}
}
        \label{rule:underapprox-refine}

        %!TEX root = ../main.tex
\infer*[lab=wp$_{\forall}$-proj,
  Right={$ I = \supp(\m{t}_1) $}
]{
  %I = \supp(\m{t}_1)
}{
  \V
  \P I. \bigl(
  \WPU{\m{t}_2}{\WPE{\m{t}_1}{\True}}
  \land
  \WPU{(\m{t}_1 \m. \m{t}_2)}{Q}
  \bigr)
  |-
  \WPU{\m{t}_2}{\PP I.Q}
}
% derivable from wp$\forall$-elim, proj-irrel, proj-weak, proj-intro
        \label{rule:wp-proj}
    \end{proofrules}
    \caption{Rules rewritten using $\wpsymb_{\exists}$}
    \label{fig:rewritten-rules}
\end{mathfig}

Using those two rewriting, one can rewrite certain rules (precisely \cref{rule:wpU-refine}, its $\wpsymb_{\exists}$ counterpart and \cref{rule:wpU-proj}) into rules combining $\wpsymb_{\forall}$ and $\wpsymb_{\exists}$. Those rules are written down in \cref{fig:rewritten-rules}.

\begin{mathfig}{\small}
    \begin{proofrules}
        %!TEX root = ../main.tex
\infer*[lab=collapse]{Det(\m{t}) \\ \WPE{\m{t}}{\True}}{
  \WPU{\m{t}}{Q} \lequiv \WPE {\m{t}} {Q}
}
        \label{rule:collapse}

        %!TEX root = ../main.tex
\infer*[lab=refine-reflexivity]{}{
  \V \pv{x}(\I 1)=\pv{x}(\I 2) |- 
        \WPU{\m{t}}{\WPE {\m{t}} {\pv{x}(\I 1)=\pv{x}(\I 2)}}
}
        \label{rule:reflexivity}

        %!TEX root = ../main.tex
\infer*[lab=proj-elim]{}{
  \V 
  \WPE {\m{t}}{\True} \implies \WPU {\m{t}} Q
  |-
  \WPU {\m{t}} Q
}

        \label{rule:proj-elim}

        %!TEX root = ../main.tex
\infer*[lab=proj-elim$_{\exists}$]{}{
  \V 
  \WPE {\m{t}}{\True}, \WPU {\m{t}} Q
  |-
  \WPE {\m{t}} Q
}

        \label{rule:proj-elim-E}
    \end{proofrules}
    \caption{General purpose rules combining $\wpsymb_{\forall}$ and $\wpsymb_{\exists}$}
    \label{fig:general-purpose}
\end{mathfig}

\subsection{Relation to $\wpsymb_{\forall}$}

We will now discuss the way to replace $\wpsymb_{\forall}$ with $\wpsymb_{\exists}$ (and vice-versa). It is of note that, assuming excluded middle, $\WPU {\m{t}} Q \lequiv \lnot(\WPE {\m{t}} {\lnot Q})$. Actually, \emph{Propositional Dynamic Logic} (PDL), introduced in \citet{Fischer79}, defines $\wpsymb_{\exists}$ in this way. However, we try to avoid non-constructivity in our logic, and we don't have a good way to reason with negation generally (mainly because of this willingness to stay constructive). We will therefore avoid this equivalence.

Still, we can define a statement regarding the two modalities:
\begin{eqnarray*}
    PDet(\m{t}) &:=& \WPE {\m{t}} Q \lequiv \WPU {\m{t}} Q
\end{eqnarray*}

This property is useful in proofs where one would like to ignore the nuance between the two modalities, and swap one with the other at will. Thankfully, this property is actually equivalent to $\m{t}$ being deterministic and projectable, two judgments that can be written inside the logic; because determinism will be used a lot, and is a bit large to write, we will use the following notation:
\begin{eqnarray*}
    Det(t) &:=& \pv{x}(\I 1) = \pv{x}(\I 2) \implies \WPU {\m[\I 1: t, \I 2: t]} {\pv{x}(\I 1) = \pv{x}(\I 2)}
\end{eqnarray*}
The notation is extended to hyperterms, by saying that an hyperterm is deterministic iff all its components are.

We will now show that $PDet(\m{t})$ is in fact \emph{equivalent} to $\proj(\m{t}) \land Det(\m{t})$

\begin{lemma}[$Det(\m{t})\land\proj(\m{t}) \iff PDet(\m{t})$]
\end{lemma} 
\begin{proof}
    If $\m{t}$ is deterministic and has a terminating trace, then any terminating trace gives the same output, given the same input. Thus, if there exists a trace that satisfies some property $Q$, then any trace will satisfy the same property. Therefore, we have the direct statement.

    For the reciprocal, we will separate the proof of $PDet(\m{t}) \implies Det(\m{t})\land\proj(\m{t})$ in two separate derivations. First off, $PDet(\m{t}) \implies \proj(\m{t})$.

    \begin{prooftree}
        \AxiomC{$\V |- \WPU {\m{t}}{\True}$}
        \AxiomC{}
        \RightLabel{\scriptsize{$PDet(t)$}}
        \UnaryInfC{$\V \WPU {\m{t}}{\True} |- \WPE {\m{t}}{\True}$}
        \BinaryInfC{$\V |- \WPE {\m{t}}{\True}$}
    \end{prooftree}

    Next, $PDet(\m{t}) \implies Det(\m{t})$. To prove it, we will show that $\A i. PDet(\m{t}(i)) \implies Det(\m{t}(i))$.
    \begin{scprooftree}{0.75}
        \AxiomC{$\V \pv{x}(\I1)=\pv{x}(\I2) |- \WPU {\m[\I1: \m{t}(i)]}{\WPE {\m[\I2: \m{t}(i)]}{\pv{x}(\I1)=\pv{x}(\I2)}}$}
        \AxiomC{}
        \RightLabel{\scriptsize{$PDet(t)$}}
        \UnaryInfC{$\V \WPE {\m[\I2: \m{t}(i)]}{\pv{x}(\I1)=\pv{x}(\I2)} |- \WPU {\m[\I2: \m{t}(i)]}{\pv{x}(\I1)=\pv{x}(\I2)}$}
        \UnaryInfC{$\V \WPU {\m[\I1: \m{t}(i)]}{\WPE {\m[\I2: \m{t}(i)]}{\pv{x}(\I1)=\pv{x}(\I2)}} |- Q$}
        \BinaryInfC{$\V \pv{x}(\I1)=\pv{x}(\I2) |- \underbrace{\WPU {\m<\I1: \m{t}(i),\I2: \m{t}(i)>}{\pv{x}(\I1)=\pv{x}(\I2)}}_Q$}
    \end{scprooftree}

    Thus we get the equivalence that we wanted.

    The left-hand side of this derivation is and instance of a important judgment : the reflexivity of the refinement relation. It will be used at length, and we chose to enshrine it as a rule: \cref{rule:reflexivity}. 
\end{proof}

\Cref{rule:collapse} gives as a rule usable in derivation this equivalence. We only give one direction of the equivalence as a rule because, as shown with the proof above, the other direction is derivable inside the logic.

\Cref{fig:general-purpose}, on top of the two rules already mentioned, gives two other rules: \cref{rule:proj-elim} and its dual \cref{rule:proj-elim-E}. The first allows to create an assumption of projectability whenever we have a $\wpsymb_{\forall}$ statement, the second allows to switch from $\wpsymb_{\exists}$ to $\wpsymb_{\forall}$, given that we prove the projectability of the term.

\subsection{Swap rule and lockstep rules}

As stated in \cref{sec:intro}, the main goal of adding the new $\wpsymb_{\exists}$ predicate was to express, and prove, hyperproperties combining both universal and existential quantifiers. Therefore, we need rules allowing us to deal with goals combining $\wpsymb_{\forall}$ and $\wpsymb_{\exists}$.

\begin{mathfig}{\small}
    \begin{proofrules}        
        %!TEX root = ../main.tex
\infer*[lab=lockstep-swap]{
  \supp(\m{t}_1)\cap\supp(\m{t}_2)=\emptyset
}{
  \V \WPE {\m{t}_2} {\WPU {\m{t}_1} {Q}}|- \WPU{\m{t}_1}{\WPE {\m{t}_2} {Q}}
}
        \label{rule:lockstep-swap}

        %!TEX root = ../main.tex
\infer*[lab=lockstep-swap-local]
{
    t_1 \semeq t_2
    \\
    PDet(t_1)\land PDet(t_2)
}{
  \V \WPE {\m[i: t_2]} {\WPU {\m[i: t_1]} {Q}}|- \WPU{\m[i: t_1]}{\WPE {\m[i: t_2]} {Q}}
}
        \label{rule:lockstep-swap-local}
    \end{proofrules}
    \caption{Swap rules}
    \label{fig:swap-rules}
\end{mathfig}

A pair of such rules are \cref{rule:lockstep-swap} and \cref{rule:lockstep-swap-local} (written in \cref{fig:swap-rules}). A special attention will be given to the first one, because it allows us to derive most lockstep rules combining $\wpsymb_{\exists}$ and $\wpsymb_{\forall}$.

Let's consider two examples: \cref{rule:lockstep-seq} and \cref{rule:lockstep-if}.

\resizebox{0.8\width}{!}{
    \begin{proofrules}
        %!TEX root = ../main.tex
\infer*[lab=wp-seq, Right={$I\cap J = \emptyset$}]{}{
    \V
    \WPU {\m[i: t_i | i \in I]}*
    {\WPEvv {\m[j: u_j | j \in J]}
    {\WPU {\m[i: t'_i | i \in I]}
    {\WPE {\m[j: u'_j | j \in J]}{Q}
    }}}
    |-
    \WPUvv {\m[i: t_i; t'_i| i \in I]}{\WPE {\m[j: u_j; u'_j | j \in J]}{Q}}
}
        \label{rule:lockstep-seq}

        %!TEX root = ../main.tex
\infer*[lab=lockstep-if, Right={$I\cap J = \emptyset$}]{}{
    \V    
    \WPUv {\m[i: g_i | i \in I]}
    {\fun b. \WPEv {\m[j: g'_j | j \in J]}
    {\fun b'. \WPUv {
        \begin{pmatrix*}[l]
        {\m[i: t_i | i\in I, \m{b}(i) \neq 0]}{\m.}\\
        {\m[i: t'_i | i\in I, \m{b}(i) = 0]}
        \end{pmatrix*}
        }{\WPE {
        \begin{pmatrix*}[l]
        {\m[i: u_j | j\in I, \m{b}'(i) \neq 0]}{\m.}\\
        {\m[i: u'_j | j\in I, \m{b}'(i) = 0]}
        \end{pmatrix*}           
        }{Q}}
    }}
    |-
    \WPUvv {\m[i: \code{if}\ g_i\ \code{then}\ t_i\ \code{else}\  t'_i| i \in I]}
    {\WPE {\m[j: \code{if}\ g'_j\ \code{then}\ u_j\ \code{else}\ u'_j | j \in J]}
    {Q}}
}
        \label{rule:lockstep-if}
    \end{proofrules}
    }

    \bigskip

    We can derive both rules from \cref{rule:lockstep-swap}, as follows:

    \bigskip

    \begin{prooftree}
        \AxiomC{$\WPU {\m[i: t_i | i \in I]}{\WPE {\m[j: u_j | j \in J]}{\WPU {\m[i: t'_i | i \in I]}{\WPE {\m[j: u'_j | j \in J]}{Q}}}}$}
        \UnaryInfC{$\WPU {\m[i: t_i | i \in I]}{\WPU {\m[j: t'_i | i \in I]}{\WPE {\m[i: u_j | j \in J]}{\WPE {\m[j: u'_j | j \in J]}{Q}}}}$}
        \UnaryInfC{$\WPU {\m[i: t_i; t'_i| i \in I]}{\WPE {\m[j: u_j; u'_j | j \in J]}{Q}}$}
    \end{prooftree}

    \bigskip

    \begin{scprooftree}{0.8}
        \AxiomC{$\WPU {\m[i: g_i | i \in I]}*
        {\fun \m{b}. \WPE {\m[j: g'_j | j \in J]}*
        {\fun \m{b}'. \WPU {
            \begin{pmatrix*}[l]
                {\m[i: t_i | i\in I, \m{b}(i) \neq 0]}{\m.}\\
                {\m[i: t'_i | i\in I, \m{b}(i) = 0]}
            \end{pmatrix*}
        }*{\WPE {
            \begin{pmatrix*}[l]
                {\m[i: u_j | j\in I, \m{b}'(i) \neq 0]}{\m.}\\
                {\m[i: u'_j | j\in I, \m{b}'(i) = 0]}
            \end{pmatrix*}           
        }{Q}}
        }}$}
        \UnaryInfC{$\WPU {\m[i: g_i | i \in I]}*
        {\fun \m{b}. \WPU {
            \begin{pmatrix*}[l]
                {\m[i: t_i | i\in I, \m{b}(i) \neq 0]}{\m.}\\
                {\m[i: t'_i | i\in I, \m{b}(i) = 0]}
            \end{pmatrix*}}*
        {\WPE {\m[j: g'_j | j \in J]
        }*{\fun \m{b}'. \WPE {
            \begin{pmatrix*}[l]
                {\m[i: u_j | j\in I, \m{b}'(i) \neq 0]}{\m.}\\
                {\m[i: u'_j | j\in I, \m{b}'(i) = 0]}
            \end{pmatrix*}           
        }{Q}}
        }}$}
        \UnaryInfC{$\WPU {\m[i: \code{if}\ g_i\ \code{then}\ t_i\ \code{else}\  t'_i| i \in I]}{\WPE {\m[j: \code{if}\ g'_j\ \code{then}\ u_j\ \code{else}\ u'_j | j \in J]}{Q}}$}
    \end{scprooftree}

    In both cases, we simply "unfold" the terms using the corresponding $\wpsymb_{\forall}$ and $\wpsymb_{\exists}$ rules, and then a number \cref{rule:lockstep-swap} to complete the proof.

    \subsection{The while rule}

    There, of course, exists a rule for \code{while} statements of the same type than for \code{if} and sequential statements.

    \begin{proofrules}
        %!TEX root = ../main.tex
\infer*[lab=wp-while, Right={$I\ne\emptyset$}]{
    \V
    P
    |- 
    \WPU {\m[i: g_i | i \in I]}*
    {\fun b'. \WPE {\m[j: g_j' | j \in J]}*
    {\begin{pmatrix*}[l]
    \fun b. (b \cdot b' =_{I\cup J} 0 \land Q) \\
    \lor 
    \begin{pmatrix*}[l]
        b \cdot b' \ne_{I\cup J} 0 \\
        \land\WPU {\m[i: t_i | i \in I]}{\WPE {\m[j: t_j' | j \in J]} {P}}
    \end{pmatrix*}
    \end{pmatrix*}}}
}{
    \V
    {P}
    |- 
    \WPU {\m[i: \code{while}\ g_i\ \code{do}\ t_i | i \in I]}
    {\WPE {\m[j: \code{while}\ g'_j\ \code{do}\ t'_j | j \in J]} {Q}}
}
        \label{rule:lockstep-while}
    \end{proofrules}

    \bigskip

    However, the proof strategy used previously doesn't work for this rule. It comes down to the fact that if we simply unfold the \code{while} loop (giving us $\code{if}\ g\ \code{then}\ \code{skip}\ \code{else}\ \code{while}\ g\ \code{do}\ t$), we can't get a terminating proof, because we get the conclusion as a premisse. The main issue is that we're missing a way to do induction proofs on the length of the loop inside the logic.

    Thankfully, there is a way to deal with this issue. We add two constructs to the language, with the goal of encoding \code{while}: \code{assume} and *. Their semantic value can be defined as follows, with $t^n$ a notation for a sequence of $n$ terms $t$.

    \begin{proofrules}
        \infer*[Left=$v\ne 0$]{\bigstep g s v {s'}}{\bigstep {\code{assume}(g)} s v {s'}}

        \infer*{\bigstep {t^n} s v {s'}}{\bigstep {t^*} s v {s'}}
    \end{proofrules}

    From this, we can encode $\code{while}\ g\ \code{do}\ t \is (\code{assume}(g); t)^*;\code{assume}(\lnot g)$. Thus, if we have rules for \code{assume} and $t^*$, we can derive the rule for \code{while}. The question is, what did we gain from this encoding? The short answer is that \code{assume} and $t^*$ have similar $\wpsymb_{\forall}-\wpsymb_{\exists}$ rules with \code{if} and sequences. The long answer is the following.

    We can derive four rules for \code{assume} and $t^*$ (described in \cref{fig:assume-*-rules}). From those rules, one can derive rules of the same form as the \code{if} one:

    \begin{proofrules}
        %!TEX root = ../main.tex
\infer*[lab=lockstep-assume]{}{
    \V \WPUvv {\m[i: g_i | i \in I]}
    {\fun \m{v}. \WPEvv {\m[j: g'_j | j \in J]} {\fun \m{w}. Q(\m{v}\cdot\m{w}) \land\m{w} \ne 0}}
    |- \WPUvv {\m[i: \code{assume}(g_i) | i \in I]}
    {\fun \m{v}. \WPEvv {\m[j: \code{assume}(g'_j) | j \in J]} {\fun \m{w}. Q(\m{v} \cdot \m{w}) \land \m{v} \ne 0}}
}
        \label{rule:lockstep-assume}

        %!TEX root = ../main.tex
\infer*[lab=lockstep-iter]{
    \V P |- 
    \WPU {\m[i: t_i | i \in I]}*
    {\WPE {\m[j: t'_j | j \in J]} P}
}{ 
    \V P |- 
    \WPU {\m[i: t_i^* | i \in I]}*
    {\WPE {\m[j: {t'_j}^* | j \in J]} P}
}
        \label{rule:lockstep-iter}
    \end{proofrules}

    \begin{mathfig}
        \begin{proofrules}
            %!TEX root = ../main.tex
\infer*[lab=wp$_{\forall}$-assume]{}{
    \V  \WPU {\m[i: g_i | i \in I]} Q |- \WPU {\m[i: \code{assume}(g_i) | i \in I]} {Q(0)}
}
            \label{rule:wpU-assume}

            %!TEX root = ../main.tex
\infer*[lab=wp$_{\exists}$-assume]{}{
    \V  \WPE {\m[i: g_i | i \in I]} {\fun r. Q(r) \land r = 0} |- \WPE {\m[i: \code{assume}(g_i) | i \in I]} Q
}
            \label{rule:wpE-assume}

            %!TEX root = ../main.tex
\infer*[lab=wp$_{\forall}$-iter]{}{
    P \land \left(\A n. \WPU {\m[i: t^n]} P \implies \WPU {\m[i: t^{n+1}]} P\right)
    \lequiv \WPU {\m[i: t_i^*]} P
}

            \label{rule:wpU-iter}

            %!TEX root = ../main.tex
\infer*[lab=wp$_{\exists}$-iter]{}{
    \WPE {\m[i: t_i^n | i \in I]} P \lequiv \WPE {\m[i: t_i^* | i \in I]} P
}
            \label{rule:wpE-iter}
        \end{proofrules}
        \caption{Rules for \code{assume} and *}
        \label{fig:assume-*-rules}
    \end{mathfig}

    They are derived as follows:

    \begin{prooftree}
        \AxiomC{$\V |- \WPU {\m[i: g_i | i \in I]}
        {\fun \m{v}. \WPE {\m[j: g'_j | j \in J]} {\fun \m{w}. Q(\m{v}\cdot\m{w}) \land\m{w} \ne 0}}$}
        \UnaryInfC{$\V |- \WPU {\m[i: \code{assume}(g_i) | i \in I]}
        {\fun \m{v}. \WPE {\m[j: g'_j | j \in J]} {\fun \m{w}. Q(\m{v} \cdot\m{w}) \land\m{w} \ne 0} \land \m{v} \ne 0}$}
        \UnaryInfC{$\V |- \WPU {\m[i: \code{assume}(g_i) | i \in I]}
        {\fun \m{v}. \WPE {\m[j: \code{assume}(g'_j) | j \in J]} {\fun \m{w}. Q(\m  {v} \cdot \m{w}) \land \m{v} \ne 0}}$}
    \end{prooftree}

    \begin{scprooftree}{0.9}
        \AxiomC{$\V P |- \WPU {\m[i: t_i]}
        {\WPE {\m[j: t'_j]} P}$}
        \AxiomC{}
        \UnaryInfC{$\A n. \V \WPU {\m[i: t_i^n]}
        {\WPE {\m[j: {t'_j}^n]} P}
        |- \WPU {\m[i: t_i^n]}
        {\WPE {\m[j: {t'_j}^n]} P}$}
        \BinaryInfC{$\A n. \V P, \WPU {\m[i: t_i^n]}
        {\WPE {\m[j: {t'_j}^n]} P}
        |- \WPU {\m[i: t_i]}
        {\WPE {\m[j: t'_j]}
        {\WPU {\m[i: t_i^n]}
        {\WPE {\m[j: {t'_j}^n]} P}}}$}
        \UnaryInfC{$\A n. \V P, \WPU {\m[i: t_i^n]}
        {\WPE {\m[j: {t'_j}^n]} P}
        |- \WPU {\m[i: t_i^{n+1}]}
        {\WPE {\m[j: {t'_j}^{n+1}]} P}$}
        \UnaryInfC{$\A n. \V P, \WPU {\m[i: t_i^n]}
        {\WPE {\m[j: {t'_j}^*]} P}
        |- \WPU {\m[i: t_i^{n+1}]}
        {\WPE {\m[j: {t'_j}^*]} P}$}
        \UnaryInfC{$\V P |- \WPU {\m[i: t_i^*]}
        {\WPE {\m[j: {t'_j}^*]} P}$}
    \end{scprooftree}

    \Cref{rule:lockstep-while} can be derived from \cref{rule:lockstep-iter} and \cref{rule:lockstep-assume} this way\footnote{It is of note that the iteration rules are currently unary, and that therefore the entire derivation only holds for two loops. It is enough for the remaining derivations of this paper, but we believe that a derivation for the more general rule exists. It remains a work in progress at time of writing}:

    \begin{scprooftree}{0.7}
        \AxiomC{$\V P |-
        \WPU {\m[i: \code{assume}(g_i)]}
        {\WPE {\m[j: \code{assume}(g'_j)]}
        {\WPU {\m[i: t_i]} {\WPE {\m[j: t'_j]} {P}}}}$}
        \UnaryInfC{$\V P |- 
        \WPU {\m[i: \code{assume}(g_i); t_i]}
        {\WPE {\m[j: \code{assume}(g'_j); t'_j]} P}$}
        \UnaryInfC{$\V P |- 
        \WPU {\m[i: (\code{assume}(g_i); t_i)^*]}
        {\WPE {\m[j: (\code{assume}(g'_j); t'_j)^*]} P}$}
        \AxiomC{$\V P |-
        \WPU {\m[i: \code{assume}(\lnot g_i)]}
        {\WPE {\m[j: \code{assume}(\lnot g'_j)]} Q}$}
        \BinaryInfC{$\V P |- 
        \WPU {\m[i: (\code{assume}(g_i); t_i)^*; \code{assume}(\lnot g_i)]}
        {\WPE {\m[j: (\code{assume}(g'_j); t'_j)^*; \code{assume}(\lnot g'_j)]} Q}$}
    \end{scprooftree}

    Both leaves can be derived from the premisse of \cref{rule:lockstep-while} by applying \cref{rule:lockstep-assume} (after negating the guards for the righthand leave), and collapsing the postcondition according to the return value being forced by the rule.

    \begin{scprooftree}{0.75}
        \AxiomC{$\V
        P
        |- 
        \WPU {\m[i: g_i]}*
        {\fun \m{b}'. \WPE {\m[j: g_j']}*
        {\begin{pmatrix*}[l]
            \fun \m{b}. (\m{b} \cdot \m{b}' =_{I\cup J} 0 \land Q) \\
            \lor 
            \begin{pmatrix*}[l]
                (\m{b} \cdot \m{b}') \ne_{I\cup J} 0 \\
                \land \WPU {\m[i: t_i]}{\WPE {\m[j: t_j']} {P}}
            \end{pmatrix*}
        \end{pmatrix*}}}$}
        \UnaryInfC{$\V
        P
        |- 
        \WPU {\m[i: \code{assume}(g_i)]}*
        {\fun \m{b}'. \WPE {\m[j: \code{assume}(g_j')]}*
        {\begin{pmatrix*}[l]
            \fun \m{b}. (\m{b} \cdot \m{b}' =_{I\cup J} 0 \land Q) \\
            \lor 
            \begin{pmatrix*}[l]
                (\m{b} \cdot \m{b}') \ne_{I\cup J} 0 \\
                \land\WPU {\m[i: t_i | i \in I]}{\WPE {\m[j: t_j']} {P}}
            \end{pmatrix*} \land (\m{b} \cdot \m{b}' \ne_{I\cup J} 0)
        \end{pmatrix*}}}$}
        \UnaryInfC{$\V
        P
        |- 
        \WPU {\m[i: \code{assume}(g_i)]}*
        {\WPE {\m[j: \code{assume}(g_j')]}*
        {\WPU {\m[i: t_i | i \in I]}{\WPE {\m[j: t_j']} {P}}}}$}
    \end{scprooftree}

    \begin{scprooftree}{0.75}
        \AxiomC{$\V
        P
        |- 
        \WPU {\m[i: g_i]}*
        {\fun \m{b}'. \WPE {\m[j: g_j']}*
        {\begin{pmatrix*}[l]
            \fun \m{b}. (\m{b} \cdot \m{b}' =_{I\cup J} 0 \land Q) \\
            \lor 
            \begin{pmatrix*}[l]
                (\m{b} \cdot \m{b}') \ne_{I\cup J} 0 \\
                \land\WPU {\m[i: t_i | i \in I]}{\WPE {\m[j: t_j']} {P}}
            \end{pmatrix*}
        \end{pmatrix*}}}$}
        \UnaryInfC{$\V
        P
        |- 
        \WPU {\m[i: \lnot g_i]}*
        {\fun \m{b}'. \WPE {\m[j: \lnot g_j']}*
        {\begin{pmatrix*}[l]
            \fun \m{b}. (\m{b} \cdot \m{b}' \ne_{I\cup J} 0 \land Q) \\
            \lor 
            \begin{pmatrix*}[l]
                (\m{b} \cdot \m{b}') =_{I\cup J} 0 \\
                \land\WPU {\m[i: t_i]}{\WPE {\m[j: t_j']} {P}}
            \end{pmatrix*}
        \end{pmatrix*}}}$}
        \UnaryInfC{$\V
        P
        |- 
        \WPU {\m[i: \code{assume}(\lnot g_i)]}*
        {\fun \m{b}'. \WPE {\m[j: \code{assume}(\lnot g_j')]}*
        {\begin{pmatrix*}[l]
            \fun \m{b}. (\m{b} \cdot \m{b}' \ne_{I\cup J} 0 \land Q) \\
            \lor 
            \begin{pmatrix*}[l]
                (\m{b} \cdot \m{b}') =_{I\cup J} 0 \\
                \land\WPU {\m[i: t_i]}{\WPE {\m[j: t_j']} {P}}
            \end{pmatrix*} \land (\m{b} \cdot \m{b}' \ne_{I\cup J} 0)
        \end{pmatrix*}}}$}
        \UnaryInfC{$\V
        P
        |- 
        \WPU {\m[i: \code{assume}(\lnot g_i)]}*
        {\WPE {\m[j: \code{assume}(\lnot g_j')]} Q}$}
    \end{scprooftree}
    
