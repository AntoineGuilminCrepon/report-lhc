%TEX root = main.tex
\section{Existential extension of LHC}
\label{sec:extension}

\subsection{New operator}

The main point of the extension is to add a modality to the language, dual to the $\wpsymb_{\forall}$ from the original model. It is defined as such :

\begin{definition}[Weakest existential precondition]
$\WPE{\m{t}}{Q} := \fun \m{s}. \E \m{s'} \m{v}. \bigstep {\m{t}}{\m{s}} {\m{v}} {\m{s'}} \land Q(\m{v})$
\end{definition}

One can easily appreciate the similarity with the original $\wpsymb_{\forall}$ definition. Intuitively, the judgment $\WPE{\m{t}}{Q}$ states that there exists a trace of $\m{t}$ satisfying $Q$. An important point is that any judgment of this form implies \emph{at least} local projectability of $\m{t}$, \ie if we chose an input state satisfying the constraint of the potential context of the judgment, $\m{t}$ must have a terminating trace beginning in this state.

When finding the rules defining the behavior of $\wpsymb_{\exists}$, it seems that most rules are direct mirrors of the $\wpsymb_{\forall}$ rules (simply replacing one predicate with the other). These "mirrored" rules are written in \cref{sec:mirrored-rules}, for completeness.

\begin{mathfig}{\small}
    \begin{proofrules}
        %!TEX root = ../main.tex
\infer*[lab=wp$_{\exists}$-exists]{}
{
  \E x.\WPE{\m{t}}{Q(x)} \lequiv \WPE{\m{t}}{\E x.Q(x)}
}
%   \exists x\st\ %
%   \V \Gamma |- \WPE{\m{t}}{Q(x)}
% }{
%   \V \Gamma |- \WPE{\m{t}}{\A x.Q(x)}
% }
        \label{rule:wpE-exists}

        %!TEX root = ../main.tex
\infer*[lab=wp$_{\exists}$-impl-l]{
  \pvar(P) \inters \mods(\m{t}) = \emptyset
}{
  \WPE{\m{t}}{\ret. Q(\ret) \implies P}
  \lequiv
  \WPE{\m{t}}{Q} \implies P
}
        \label{rule:wpE-impl-l}

        %!TEX root = ../main.tex
\infer*[lab=wp$_{\exists}$-while$_I$]{
  \V
  {\A \alpha. P(\alpha)}
  |-
  \WPE {\m[i: g_i | i\in I]}[\big]{
    \fun\m{b}.
    (\m{b} =_I 0 \land R)
    \lor
    (\m{b} \ne_I 0 \land \WPE {\m[i: t_i | i \in I]} {\E \alpha'. {P(\alpha') \land \alpha' < \alpha}})
  }
}{
  \V
  {P (\alpha_0)}
  |-
  \WPE {\m[i: \code{while}\ g_i\ \code{do}\ t_i | i \in I]} {R}
}
        \label{rule:wpE-while}

        %!TEX root = ../main.tex
\infer*[lab=wp$_{\exists}$-conj]{
  % \idx(Q_1) \subs \supp(\m{t}_1)
  \idx(Q_1) \inters \supp(\m{t}_2) \subs \supp(\m{t}_1)
  \\
  % \idx(Q_2) \subs \supp(\m{t}_2)
  \idx(Q_2) \inters \supp(\m{t}_1) \subs \supp(\m{t}_2)
}{
  \V
  \WPE{\m{t}_1}{Q_1}
  \land
  \WPE{\m{t}_2}{Q_2}
  |-
  \WPE{(\m{t}_1 \m+ \m{t}_2)}{Q_1 \land Q_2}
}

        \label{rule:wpE-conj}
    \end{proofrules}
    \caption{Rules specific to $\wpsymb_{\exists}$}
    \label{fig:wpE-rules}
\end{mathfig}

\Cref{fig:wpE-rules} gives out the rules applying to $\wpsymb_{\exists}$ that are substantially different from their $\wpsymb_{\forall}$ couterparts. \Cref{rule:wpE-exists,rule:wpE-impl-l} show that $\wpsymb_{\exists}$ commutes with $\exists$ and $(\implies P)$ (note the duality with \cref{rule:wpU-all,rule:wpU-impl-r}). Because any $\wpsymb_{\exists}$ judgment implies the projectability of the hyperterm, the "invariant" $P$ in \cref{rule:wpE-while} needs to be parametrized by a decreasing loop index $\alpha$. This loop index needs of course to be in a well-founded order (most of the time, natural numbers should suffice, but the possibility to call on transfinite ordinals reinforces the rule). \Cref{rule:wpE-conj} is slightly weaker than its $\wpsymb_{\forall}$ sibling, because the two hyperterms now requires to be disjoint. The reason is that the two traces given by the premisse $\WPE{\m{t}_1}{Q_1} \land \WPE{\m{t}_2}{Q_2}$ do not require to coincide on $\m{t}_1\cap\m{t}_2$

An interesting point is that predicates that couldn't be defined in the original theory using the base langugage, and thus needed to be explicitly defined to be used, can now be defined using this new $\wpsymb_{\exists}$ object.

\begin{eqnarray*}
    \proj(t) &:=& \WPE{\m{t}}{\True} \\
    t_1 \semleq t_2 &:=& \V s(1)=s(2) |- \WPU{[1: t_1]}{\WPE{[2: t_2]}{v(1) = v(2) \land s(1) = s(2)}}
\end{eqnarray*}

Thus certain rules from LHC can now be rewritten by combining $\wpsymb_{\forall}$ and $\wpsymb_{\exists}$

\begin{mathfig}{\small}
    \begin{proofrules}
        %!TEX root = ../main.tex
\infer*[lab=overapprox-refine]{}{
  \V
  \pv{x}(\I 1) = \pv{x}(\I 2) \implies \WPU{[1: t_1]}{\WPE{[2: t_2]}{\pv{x}(\I 1) = \pv{x}(\I 2)}},
  \WPU {(\m*[i: t_2] \m. \m{t})} {Q}
  |-
  \WPU {(\m[i: t_1] \m. \m{t})} {Q}
}
        \label{rule:overapprox-refine}

        %!TEX root = ../main.tex
\infer*[lab=underapprox-refine]{}{
  \V
  \pv{x}(\I 1) = \pv{x}(\I 2) \implies \WPU{[1: t_1]}{\WPE{[2: t_2]}{\pv{x}(\I 1) = \pv{x}(\I 2)}},
  \WPE {(\m*[i: t_1] \m. \m{t})} {Q}
  |-
  \WPE {(\m[i: t_2] \m. \m{t})} {Q}
}
        \label{rule:underapprox-refine}

        %!TEX root = ../main.tex
\infer*[lab=wp$_{\forall}$-proj,
  Right={$ I = \supp(\m{t}_1) $}
]{
  %I = \supp(\m{t}_1)
}{
  \V
  \P I. \bigl(
  \WPU{\m{t}_2}{\WPE{\m{t}_1}{\True}}
  \land
  \WPU{(\m{t}_1 \m. \m{t}_2)}{Q}
  \bigr)
  |-
  \WPU{\m{t}_2}{\PP I.Q}
}
% derivable from wp$\forall$-elim, proj-irrel, proj-weak, proj-intro
        \label{rule:wp-proj}
    \end{proofrules}
    \caption{Rules rewritten using $\wpsymb_{\exists}$}
\end{mathfig}

\begin{mathfig}{\small}
    \begin{proofrules}
        %!TEX root = ../main.tex
\infer*[lab=lockstep-triv]{Det(\m{t}) \\ \WPE{\m{t}}{\True}}{
  \WPU{\m{t}}{Q} \lequiv \WPE {\m{t}} {Q}
}
        \label{rule:lockstep-triv}

        %!TEX root = ../main.tex
\infer*[lab=proj-elim]{}{
  \V 
  \WPE {\m{t}}{\True} \implies \WPU {\m{t}} Q
  |-
  \WPU {\m{t}} Q
}

        \label{rule:proj-elim}
    \end{proofrules}
    \caption{Rules combining $\wpsymb_{\forall}$ and $\wpsymb_{\exists}$}
\end{mathfig}

\subsection{Relation to $\wpsymb_{\forall}$}

With the use of $\wpsymb_{\exists}$, one can write an interesting property:
\begin{eqnarray*}
    PDet(t) &:=& \WPE {\m{t}} Q \lequiv \WPU {\m{t}} Q
\end{eqnarray*}

This property is in fact equivalent to $Det(t) \land \proj(t)$, as is shown below.

\begin{lemma}[$Det(t)\land\proj(t) \iff PDet(t)$]
\end{lemma} 
\begin{proof}
    If $\m{t}$ is deterministic and has a terminating trace, then any terminating trace gives the same output, given the same input. Thus, if there exists a trace that satisfies some property $Q$, then any trace will satisfy the same property. Therefore, we have the direct statement.

    For the reciprocal, we will separate the proof of $PDet(t) \implies Det(t)\land\proj(t)$ in two separate derivations. First off, $PDet(t) \implies \proj(t)$.

    \begin{prooftree}
        \AxiomC{}
        \LeftLabel{\scriptsize\textsf{wp$_{\forall}$-triv}}
        \UnaryInfC{$\V |- \WPU {\m{t}}{\True}$}
        \AxiomC{}
        \RightLabel{\scriptsize{$PDet(t)$}}
        \UnaryInfC{$\V \WPU {\m{t}}{\True} |- \WPE {\m{t}}{\True}$}
        \BinaryInfC{$\V |- \WPE {\m{t}}{\True}$}
    \end{prooftree}

    Next, $PDet(t) \implies Det(t)$.
    \begin{scprooftree}{0.9}
        \AxiomC{$\V \pv{x}(\I1)=\pv{x}(\I2) |- \WPE {\m<\I1: t, \I2: t>}{\pv{x}(\I1)=\pv{x}(\I2)}$}
        \AxiomC{}
        \RightLabel{\scriptsize{$PDet(t)$}}
        \UnaryInfC{$\V \WPE {\m<\I1: t, \I2: t>}{\pv{x}(\I1)=\pv{x}(\I2)} |- \WPU {\m<\I1: t, \I2: t>}{\pv{x}(\I1)=\pv{x}(\I2)}$}
        \BinaryInfC{$\V \pv{x}(\I1)=\pv{x}(\I2) |- \WPU {\m<\I1: t,\I2: t>}{\pv{x}(\I1)=\pv{x}(\I2)}$}
    \end{scprooftree}

    The left-side assumption is proved the following way: given that both terms receive the same input, if we chose the same trace to execute in both terms, we then get the same output (\ie every term is "deterministic" in the context of $\wpsymb_{\exists}$).

    Thus we get an equivalence between $PDet(t)$ and $Det(t)\land\proj(t)$
\end{proof}

\subsection{Swap rule and lockstep rules}

As stated before, $\wpsymb_{\exists}$ behaves more or less the same way as $\wpsymb_{\forall}$, and thus, proofs using it exclusively are very similar in form. Most of the capabilities of this model appears when combining both operators. However, one would need rules to deal with hyperproperties combining the two $\wpsymb$.

\begin{mathfig}{\small}
    \begin{proofrules}        
        %!TEX root = ../main.tex
\infer*[lab=lockstep-swap]{
  \supp(\m{t}_1)\cap\supp(\m{t}_2)=\emptyset
}{
  \V \WPE {\m{t}_2} {\WPU {\m{t}_1} {Q}}|- \WPU{\m{t}_1}{\WPE {\m{t}_2} {Q}}
}
        \label{rule:lockstep-swap}

        %!TEX root = ../main.tex
\infer*[lab=lockstep-swap-local]
{
    t_1 \semeq t_2
    \\
    PDet(t_1)\land PDet(t_2)
}{
  \V \WPE {\m[i: t_2]} {\WPU {\m[i: t_1]} {Q}}|- \WPU{\m[i: t_1]}{\WPE {\m[i: t_2]} {Q}}
}
        \label{rule:lockstep-swap-local}
    \end{proofrules}
    \caption{Swap rules}
\end{mathfig}

A pair of such rules are \cref{rule:lockstep-swap} and \cref{rule:lockstep-swap-local}. A special attention will be given to the first one, because it allows us to derive most lockstep rules combining $\wpsymb_{\exists}$ and $\wpsymb_{\forall}$.

Let's consider two examples: \cref{rule:lockstep-if} and \cref{rule:lockstep-seq}.

\resizebox{0.8\width}{!}{
\begin{proofrules}
    %!TEX root = ../main.tex
\infer*[lab=wp-seq, Right={$I\cap J = \emptyset$}]{}{
    \V
    \WPU {\m[i: t_i | i \in I]}*
    {\WPEvv {\m[j: u_j | j \in J]}
    {\WPU {\m[i: t'_i | i \in I]}
    {\WPE {\m[j: u'_j | j \in J]}{Q}
    }}}
    |-
    \WPUvv {\m[i: t_i; t'_i| i \in I]}{\WPE {\m[j: u_j; u'_j | j \in J]}{Q}}
}
    \label{rule:lockstep-seq}

    %!TEX root = ../main.tex
\infer*[lab=lockstep-if, Right={$I\cap J = \emptyset$}]{}{
    \V    
    \WPUv {\m[i: g_i | i \in I]}
    {\fun b. \WPEv {\m[j: g'_j | j \in J]}
    {\fun b'. \WPUv {
        \begin{pmatrix*}[l]
        {\m[i: t_i | i\in I, \m{b}(i) \neq 0]}{\m.}\\
        {\m[i: t'_i | i\in I, \m{b}(i) = 0]}
        \end{pmatrix*}
        }{\WPE {
        \begin{pmatrix*}[l]
        {\m[i: u_j | j\in I, \m{b}'(i) \neq 0]}{\m.}\\
        {\m[i: u'_j | j\in I, \m{b}'(i) = 0]}
        \end{pmatrix*}           
        }{Q}}
    }}
    |-
    \WPUvv {\m[i: \code{if}\ g_i\ \code{then}\ t_i\ \code{else}\  t'_i| i \in I]}
    {\WPE {\m[j: \code{if}\ g'_j\ \code{then}\ u_j\ \code{else}\ u'_j | j \in J]}
    {Q}}
}
    \label{rule:lockstep-if}
\end{proofrules}
}

\bigskip

We can derive both rules from \cref{rule:lockstep-swap}, as follows:

\bigskip

\begin{prooftree}
    \AxiomC{$\WPU {\m[i: t_i | i \in I]}{\WPE {\m[j: u_j | j \in J]}{\WPU {\m[i: t'_i | i \in I]}{\WPE {\m[j: u'_j | j \in J]}{Q}}}}$}
    \UnaryInfC{$\WPU {\m[i: t_i | i \in I]}{\WPU {\m[j: t'_i | i \in I]}{\WPE {\m[i: u_j | j \in J]}{\WPE {\m[j: u'_j | j \in J]}{Q}}}}$}
    \UnaryInfC{$\WPU {\m[i: t_i; t'_i| i \in I]}{\WPE {\m[j: u_j; u'_j | j \in J]}{Q}}$}
\end{prooftree}

\bigskip

\begin{scprooftree}{0.8}
    \AxiomC{$\WPU {\m[i: g_i | i \in I]}*
    {\fun \m{b}. \WPE {\m[j: g'_j | j \in J]}*
    {\fun \m{b}'. \WPU {
        \begin{pmatrix*}[l]
        {\m[i: t_i | i\in I, \m{b}(i) \neq 0]}{\m.}\\
        {\m[i: t'_i | i\in I, \m{b}(i) = 0]}
        \end{pmatrix*}
        }*{\WPE {
        \begin{pmatrix*}[l]
        {\m[i: u_j | j\in I, \m{b}'(i) \neq 0]}{\m.}\\
        {\m[i: u'_j | j\in I, \m{b}'(i) = 0]}
        \end{pmatrix*}           
        }{Q}}
    }}$}
    \UnaryInfC{$\WPU {\m[i: g_i | i \in I]}*
    {\fun \m{b}. \WPU {
        \begin{pmatrix*}[l]
        {\m[i: t_i | i\in I, \m{b}(i) \neq 0]}{\m.}\\
        {\m[i: t'_i | i\in I, \m{b}(i) = 0]}
        \end{pmatrix*}}*
        {\WPE {\m[j: g'_j | j \in J]
        }*{\fun \m{b}'. \WPE {
        \begin{pmatrix*}[l]
        {\m[i: u_j | j\in I, \m{b}'(i) \neq 0]}{\m.}\\
        {\m[i: u'_j | j\in I, \m{b}'(i) = 0]}
        \end{pmatrix*}           
        }{Q}}
    }}$}
    \UnaryInfC{$\WPU {\m[i: \code{if}\ g_i\ \code{then}\ t_i\ \code{else}\  t'_i| i \in I]}{\WPE {\m[j: \code{if}\ g'_j\ \code{then}\ u_j\ \code{else}\ u'_j | j \in J]}{Q}}$}
\end{scprooftree}

In both cases, we simply "unfold" the terms using the corresponding $\wpsymb_{\forall}$ and $\wpsymb_{\exists}$ rules, and then a number \cref{rule:lockstep-swap} to complete the proof.

\subsection{The while rule}

There, of course, exists a rule for \code{while} statements of the same type than for \code{if} and sequential statements.

\begin{proofrules}
    %!TEX root = ../main.tex
\infer*[lab=wp-while, Right={$I\ne\emptyset$}]{
    \V
    P
    |- 
    \WPU {\m[i: g_i | i \in I]}*
    {\fun b'. \WPE {\m[j: g_j' | j \in J]}*
    {\begin{pmatrix*}[l]
    \fun b. (b \cdot b' =_{I\cup J} 0 \land Q) \\
    \lor 
    \begin{pmatrix*}[l]
        b \cdot b' \ne_{I\cup J} 0 \\
        \land\WPU {\m[i: t_i | i \in I]}{\WPE {\m[j: t_j' | j \in J]} {P}}
    \end{pmatrix*}
    \end{pmatrix*}}}
}{
    \V
    {P}
    |- 
    \WPU {\m[i: \code{while}\ g_i\ \code{do}\ t_i | i \in I]}
    {\WPE {\m[j: \code{while}\ g'_j\ \code{do}\ t'_j | j \in J]} {Q}}
}
    \label{rule:lockstep-while}
\end{proofrules}

\bigskip

However, the proof strategy used previously doesn't work for this rule. It comes down to the fact that if we simply unfold \code{while} ($\code{while}\ g\ \code{do}\ t \implies \code{if}\ g\ \code{then}\ \code{skip}\ \code{else}\ \code{while}\ g\ \code{do}\ t$), we can't get a terminating proof, because we get the conclusion as a premisse. The main issue is that we're missing a way to do induction proofs on the length of the loop inside the logic.

Thankfully, there may be a way to deal with this issue, using a certain encoding of \code{while}. We will add to construct to the language: \code{assume} and *. Their semantic value can be defined as follows, with $t^n$ a notation for a sequence of $n$ terms $t$.

\begin{proofrules}
    \infer*{\E v\ne 0. \bigstep g s v {s'}}{\bigstep {\code{assume}(g)} s 1 {s'}}

    \infer*{\E n. \bigstep {t^n} s v {s'}}{\bigstep {t^*} s v {s'}}
\end{proofrules}

From this, we can encode $\code{while}\ g\ \code{do}\ t \is (\code{assume}(g); t)^*;\code{assume}(\lnot g)$. Thus, if we have rules for \code{assume} and *, we can derive the rule for \code{while}. The question is, what did we gain from this encoding ? The short answer is that \code{assume} and * have similar $\wpsymb_{\forall}-\wpsymb_{\exists}$ rules with \code{if} and sequences. The long answer is the following.

We can derive four rules for \code{assume} and * (described in \cref{fig:assume-*-rules}). From those rules, one can derive rules of the same form as the \code{if} one:

\begin{proofrules}
    %!TEX root = ../main.tex
\infer*[lab=lockstep-assume]{}{
    \V \WPUvv {\m[i: g_i | i \in I]}
    {\fun \m{v}. \WPEvv {\m[j: g'_j | j \in J]} {\fun \m{w}. Q(\m{v}\cdot\m{w}) \land\m{w} \ne 0}}
    |- \WPUvv {\m[i: \code{assume}(g_i) | i \in I]}
    {\fun \m{v}. \WPEvv {\m[j: \code{assume}(g'_j) | j \in J]} {\fun \m{w}. Q(\m{v} \cdot \m{w}) \land \m{v} \ne 0}}
}
    \label{rule:lockstep-assume}

    %!TEX root = ../main.tex
\infer*[lab=wp-star]{
    \V P |- 
    \WPU {\m[i: t_i | i \in I]}*
    {\WPE {\m[j: t'_j | j \in J]} P}
}{ 
    \V P |- 
    \WPU {\m[i: t_i^* | i \in I]}*
    {\WPE {\m[j: {t'_j}^* | j \in J]} P}
}
    \label{rule:lockstep-star}
\end{proofrules}

\begin{mathfig}
    \begin{proofrules}
        %!TEX root = ../main.tex
\infer*[lab=wp$_{\forall}$-assume]{}{
    \V  \WPU {\m[i: g_i | i \in I]} Q |- \WPU {\m[i: \code{assume}(g_i) | i \in I]} {Q(0)}
}
        \label{rule:wpU-assume}

        %!TEX root = ../main.tex
\infer*[lab=wp$_{\exists}$-assume]{}{
    \V  \WPE {\m[i: g_i | i \in I]} {\fun r. Q(r) \land r = 0} |- \WPE {\m[i: \code{assume}(g_i) | i \in I]} Q
}
        \label{rule:wpE-assume}

        %!TEX root = ../main.tex
\infer*[lab=wp$_{\forall}$-star]{}{
    \V  P \land \left(\A n. \WPU {\m[i: t^n | i \in I]} P \implies \WPU {\m[i: t^{n+1} | i \in I]} P\right)
    |- \WPU {\m[i: t_i^* | i \in I]} P
}
        \label{rule:wpU-star}

        %!TEX root = ../main.tex
\infer*[lab=wp$_{\exists}$-star]{}{
    \V  \WPE {\m[i: t_i^n | i \in I]} P |- \WPE {\m[i: t_i^* | i \in I]} P
}
        \label{rule:wpE-star}
    \end{proofrules}
    \caption{Rules for \code{assume} and *}
    \label{fig:assume-*-rules}
\end{mathfig}

They are derived as follows:

\begin{prooftree}
    \AxiomC{$\WPU {\m[i: g_i | i \in I]}
            {\fun \m{v}. \WPE {\m[j: g'_j | j \in J]} {\fun \m{w}. Q(\m{v}\cdot\m{w}) \land\m{w} \ne 0}}$}
    \UnaryInfC{$\WPU {\m[i: \code{assume}(g_i) | i \in I]}
                {\WPE {\m[j: g'_j | j \in J]} {\fun \m{w}. Q(1\cdot\m{w}) \land\m{w} \ne 0}}$}
    \UnaryInfC{$\WPU {\m[i: \code{assume}(g_i) | i \in I]}
                {\WPE {\m[j: \code{assume}(g'_j) | j \in J]} {\fun \m{w}. Q(1 \cdot \m{w})}}$}
\end{prooftree}

\begin{scprooftree}{0.75}
    \AxiomC{$\V P |- \WPU {\m[i: t_i | i \in I]}
            {\WPE {\m[j: t'_j | j \in J]} P}$}
    \AxiomC{}
    \UnaryInfC{$\A n. \V \WPU {\m[i: t_i^n | i \in I]}
            {\WPE {\m[j: {t'_j}^n | j \in J]} P}
            |- \WPU {\m[i: t_i^n | i \in I]}
            {\WPE {\m[j: {t'_j}^n | j \in J]} P}$}
    \BinaryInfC{$\A n. \V P, \WPU {\m[i: t_i^n | i \in I]}
                {\WPE {\m[j: {t'_j}^n | j \in J]} P}
                |- \WPU {\m[i: t_i | i \in I]}
                {\WPE {\m[j: t'_j | j \in J]}
                {\WPU {\m[i: t_i^n | i \in I]}
                {\WPE {\m[j: {t'_j}^n | j \in J]} P}}}$}
    \UnaryInfC{$\A n. \V P, \WPU {\m[i: t_i^n | i \in I]}
                {\WPE {\m[j: {t'_j}^n | j \in J]} P}
                |- \WPU {\m[i: t_i^{n+1} | i \in I]}
                {\WPE {\m[j: {t'_j}^{n+1} | j \in J]} P}$}
    \UnaryInfC{$\A n. \V P, \WPU {\m[i: t_i^n | i \in I]}
                {\WPE {\m[j: {t'_j}^* | j \in J]} P}
                |- \WPU {\m[i: t_i^{n+1} | i \in I]}
                {\WPE {\m[j: {t'_j}^* | j \in J]} P}$}
    \UnaryInfC{$\V P |- \WPU {\m[i: t_i^* | i \in I]}
                {\WPE {\m[j: {t'_j}^* | j \in J]} P}$}
\end{scprooftree}

\Cref{rule:lockstep-while} can be derived from \cref{rule:lockstep-star} and \cref{rule:lockstep-assume} this way:

\begin{scprooftree}{0.7}
    \AxiomC{$\V P |-
                \WPU {\m[i: \code{assume}(g_i)]}
                {\WPE {\m[j: \code{assume}(g'_j)]}
                {\WPU {\m[i: t_i]} {\WPE {\m[j: t'_j]} {P}}}}$}
    \UnaryInfC{$\V P |- 
                \WPU {\m[i: \code{assume}(g_i); t_i]}
                {\WPE {\m[j: \code{assume}(g'_j); t'_j]} P}$}
    \UnaryInfC{$\V P |- 
                \WPU {\m[i: (\code{assume}(g_i); t_i)^*]}
                {\WPE {\m[j: (\code{assume}(g'_j); t'_j)^*]} P}$}
    \AxiomC{$\V P |-
                \WPU {\m[i: \code{assume}(\lnot g_i)]}
                {\WPE {\m[j: \code{assume}(\lnot g'_j)]} Q}$}
    \BinaryInfC{$\V P |- 
                \WPU {\m[i: (\code{assume}(g_i); t_i)^*; \code{assume}(\lnot g_i)]}
                {\WPE {\m[j: (\code{assume}(g'_j); t'_j)^*; \code{assume}(\lnot g'_j)]} Q}$}
\end{scprooftree}

Both leaves can be derived from the premisse of \cref{rule:lockstep-while} by applying \cref{rule:lockstep-assume} (after negating the guards for the righthand leave), and collapsing the postcondition according to the return value being forced by the rule.

\begin{scprooftree}{0.75}
    \AxiomC{$\V
    P
    |- 
    \WPU {\m[i: g_i | i \in I]}*
    {\fun \m{b}'. \WPE {\m[j: g_j' | j \in J]}*
    {\begin{pmatrix*}[l]
    \fun \m{b}. (\m{b} \cdot \m{b}' =_{I\cup J} 0 \land Q) \\
    \lor 
    \begin{pmatrix*}[l]
        (\m{b} \cdot \m{b}') \ne_{I\cup J} 0 \\
        \land \WPU {\m[i: t_i | i \in I]}{\WPE {\m[j: t_j' | j \in J]} {P}}
    \end{pmatrix*}
    \end{pmatrix*}}}$}
    \UnaryInfC{$\V
    P
    |- 
    \WPU {\m[i: \code{assume}(g_i) | i \in I]}*
    {\fun \m{b}'. \WPE {\m[j: \code{assume}(g_j') | j \in J]}*
    {\begin{pmatrix*}[l]
    \fun \m{b}. (\m{b} \cdot \m{b}' =_{I\cup J} 0 \land Q) \\
    \lor 
    \begin{pmatrix*}[l]
        (\m{b} \cdot \m{b}') \ne_{I\cup J} 0 \\
        \land\WPU {\m[i: t_i | i \in I]}{\WPE {\m[j: t_j' | j \in J]} {P}}
    \end{pmatrix*} \land (\m{b} \cdot \m{b}' \ne_{I\cup J} 0)
    \end{pmatrix*}}}$}
    \UnaryInfC{$\V
    P
    |- 
    \WPU {\m[i: \code{assume}(g_i) | i \in I]}*
    {\WPE {\m[j: \code{assume}(g_j') | j \in J]}*
    {\WPU {\m[i: t_i | i \in I]}{\WPE {\m[j: t_j' | j \in J]} {P}}}}$}
\end{scprooftree}

\begin{scprooftree}{0.75}
    \AxiomC{$\V
    P
    |- 
    \WPU {\m[i: g_i | i \in I]}*
    {\fun \m{b}'. \WPE {\m[j: g_j' | j \in J]}*
    {\begin{pmatrix*}[l]
    \fun \m{b}. (\m{b} \cdot \m{b}' =_{I\cup J} 0 \land Q) \\
    \lor 
    \begin{pmatrix*}[l]
        (\m{b} \cdot \m{b}') \ne_{I\cup J} 0 \\
        \land\WPU {\m[i: t_i | i \in I]}{\WPE {\m[j: t_j' | j \in J]} {P}}
    \end{pmatrix*}
    \end{pmatrix*}}}$}
    \UnaryInfC{$\V
    P
    |- 
    \WPU {\m[i: \lnot g_i | i \in I]}*
    {\fun \m{b}'. \WPE {\m[j: \lnot g_j' | j \in J]}*
    {\begin{pmatrix*}[l]
    \fun \m{b}. (\m{b} \cdot \m{b}' \ne_{I\cup J} 0 \land Q) \\
    \lor 
    \begin{pmatrix*}[l]
        (\m{b} \cdot \m{b}') =_{I\cup J} 0 \\
        \land\WPU {\m[i: t_i | i \in I]}{\WPE {\m[j: t_j' | j \in J]} {P}}
    \end{pmatrix*}
    \end{pmatrix*}}}$}
    \UnaryInfC{$\V
    P
    |- 
    \WPU {\m[i: \code{assume}(\lnot g_i) | i \in I]}*
    {\fun \m{b}'. \WPE {\m[j: \code{assume}(\lnot g_j') | j \in J]}*
    {\begin{pmatrix*}[l]
    \fun \m{b}. (\m{b} \cdot \m{b}' \ne_{I\cup J} 0 \land Q) \\
    \lor 
    \begin{pmatrix*}[l]
        (\m{b} \cdot \m{b}') =_{I\cup J} 0 \\
        \land\WPU {\m[i: t_i | i \in I]}{\WPE {\m[j: t_j' | j \in J]} {P}}
    \end{pmatrix*} \land (\m{b} \cdot \m{b}' \ne_{I\cup J} 0)
    \end{pmatrix*}}}$}
    \UnaryInfC{$\V
    P
    |- 
    \WPU {\m[i: \code{assume}(\lnot g_i) | i \in I]}*
    {\WPE {\m[j: \code{assume}(\lnot g_j') | j \in J]} Q}$}
\end{scprooftree}