%!TEX root = main.tex
\section{Presentation of LHC}

This section is inspired by the paper presenting LHC. %insert link here%
The logic defines, upon an arbitrary toy language, hyperterms, hyperstores, hyperreturn values and hyperproperties.

For the rest of the paper, $\PVar$ is the set of variables, $\Val\is\Int$ the set of values and $\Idx\is\Nat$ the set of indexes for hyperobjects.

For all sets $A, B$, one notes $A \pto B$ a partial function from $A$ to $B$, i.e a function of type $A\rightarrow B_{\bot}$, where $B_{\bot} = B\cup\{\bot\}$.

Then comes the following definitions ($\Store$ is the set of stores, $\Term$ the set of terms in the language).

\begin{definition}
\begin{grammar}
  t \in\Term \is x \mid v \mid * \mid e_1 \oplus e_2 \mid \code{if}\ g\ \code{then}\ t1\ \code{else}\ t_2 \mid \code{while}\ g\ \code{do}\ t \\
\end{grammar}
\begin{eqnarray*}
  \Store&\is&\PVar\rightarrow\Val \\
  \Type{HStore}&\is&\Idx\rightarrow\Store \\
  \Type{HReturn}&\is&\Idx\rightarrow\Val \\
  \Assrt&\is&\Store\rightarrow\Prop \\
  \HAssrt&\is&HStore\rightarrow\Prop \\
  \Type{PostHAssrt}&\is&\Type{HReturn}\rightarrow\HAssrt \\
\end{eqnarray*}
\end{definition}

Furthermore, we define a reindexing as any (potentially non-bijective) function of type $\Idx\rightarrow\Idx$. For any reindexing $\pi$ and any $f\of\Idx\rightarrow A$, one notes $f\isub*{\pi} = \fun i. f(\pi(i))$ the reindexing of $f$ through $\pi$.

Fig. 1 gives notations for hyperassertions that are used later in the paper. Special attention needs to be brought to the $\Pi$ operator, which is the least intuitive. It is semantically a reduction of a given hyperassertion, by reducing its scope to the given set of indexes $I$.

\begin{mathfig}[\small]
  \begin{align*}
    (\p{x}(i) = v) &\is
      \fun \m{s}.
        \m{s}(i)(\p{x}) = v
      % && \text{Indexed program variable}
    &
    P_1 \implies P_2 &\is
      \fun \m{s}.
        P_1(\m{s}) \implies P_2(\m{s})
      % && \text{Implication}
    \\
    P_1 \land P_2 &\is
      \fun \m{s}. P_1(\m{s}) \land P_2(\m{s})
      % && \text{Conjunction}
    &
    P_1 \lor P_2 &\is
      \fun \m{s}. P_1(\m{s}) \lor P_2(\m{s})
      % && \text{Disjunction}
    \\
    \E x.P(x) &\is
      \fun \m{s}.\exists x\st P(x)(\m{s})
      % && \text{Existential quantification}
    &
    \A x.P(x) &\is
      \fun \m{s}.\forall x\st P(x)(\m{s})
      % && \text{Universal quantification}
    \\
    P\isub*{\pi} &\is
      \fun \m{s}.
        % P(\m[i: \m{s}(\pi(i)) | i \in \Idx])
        P(\m{s}\isub*{\pi})
    &
    Q\isub*{\pi} &\is
      \ret.Q(\ret\isub*{\pi})\isub*{\pi}
      % && \text{Index substitution}
    \\
    \P I.P &\is
      \fun \m{s}.
        \exists \pr{\m{s}}\st
          P(\m{s}\m[i: \m{s}'\!(i) | i \in I])
    &
    \PP I.Q &\is \ret.\E \m{v}.\P I.Q(\ret\m[i: \m{v}(i) | i\in I])
      % && \text{Projection}
    \\
    \at{A}{I} &\is
      \fun \m{s}. \LAnd_{i\in I} A(\m{s}(i))
      % && \text{Assertion lifting}
    &
    Q_1 \land Q_2 &\is \ret. Q_1(\ret) \land Q_2(\ret)
    % \\
    % \pure{\phi} &\is \fun \wtv.\phi
    % &
    % \T{P}{\m{t}}{Q} &\is \pure*{\J |- {P}{\m{t}}{Q}}
   \end{align*}
  \caption{Hyper-assertions}
  \label{fig:hyper-assertions}
\end{mathfig}

We now introduces a new object, directly inspired from the original Hoare logic: the weakest precondition operator. Extending it to hyperterms and hyperassertions gives the following definition:

\begin{definition}[Weakest precondition]
  $\WPU{\m{t}}{Q} ::= \lambda s. \A s' v. \bigstep t s v {s'} \implies Q(v)$
\end{definition}



\begin{mathfig}[\small]
  \begin{proofrules}
    %!TEX root = ../main.tex
\infer*[lab=wp$_{\forall}$-triv]{}{
  \V |- \WPU{\m{t}}{\True}
}

    %!TEX root = ../main.tex
\infer*[lab=wp$_{\forall}$-cons]{
  \forall \m{v}\st\ %
  \V Q(\m{v}) |- Q'(\m{v})
}{
  \V \WPU{\m{t}}{Q} |- \WPU{\m{t}}{Q'}
}

    %!TEX root = ../main.tex
\infer*[lab=wp$_{\forall}$-all]{}
{
  \A x.\WPU{\m{t}}{Q(x)} \lequiv \WPU{\m{t}}{\A x.Q(x)}
}
%   \forall x\st\ %
%   \V \Gamma |- \WPU{\m{t}}{Q(x)}
% }{
%   \V \Gamma |- \WPU{\m{t}}{\A x.Q(x)}
% }

    %!TEX root = ../main.tex
\infer*[lab=wp$_{\forall}$-frame]{
  \pvar(P) \inters \mods(\m{t}) = \emptyset
}{
  \V P \land \WPU{\m{t}}{Q} |- \WPU {\m{t}} {P \land Q}
}

    %!TEX root = ../main.tex
\infer*[lab=wp$_{\forall}$-impl-r]{
  \pvar(P) \inters \mods(\m{t}) = \emptyset
}{
  P \implies \WPU{\m{t}}{Q}
  \lequiv
  \WPU{\m{t}}{\ret. P \implies Q(\ret)}
}

    %!TEX root = ../main.tex
\infer*[lab=wp$_{\forall}$-subst]{
  \p{x} \notin \mods(t)
}{
  \V
  \p{x}(i) = v
  \land
  \WPU{\bigl(\map[\big]{i: t\subst{\p{x}->v}}\m.\m{t}'\bigr)}{Q}
  |-
  \WPU{(\m[i: t]\m.\m{t}')}{Q}
}


    %!TEX root = ../main.tex
\infer*[lab=wp$_{\forall}$-idx]{
  \pi\ \mathrm{bijective}
}{
  \V (\WPU{\m{t}}{Q})\isub*{\pi} |- \WPU{\m{t}\isub*{\pi}}{Q\isub*{\pi}}
}
  \end{proofrules}
  \caption{Base rules for $\mathbf{wp}_{\forall}$ from LHC}
\end{mathfig}

\begin{mathfig}{\small}
  \begin{proofrules}
    %!TEX root = ../main.tex
\infer*[lab=wp$_{\forall}$-var]{}{
  \V |- \WPU { \m[i: \p{x}] } { \ret. \ret(i) = \p{x}(i) }
}

    %!TEX root = ../main.tex
\infer*[lab=wp$_{\forall}$-val]{}{
  \V |- \WPU {\m[i: v]}{\ret.\ret[i] = v}
}

    %!TEX root = ../main.tex
\infer*[lab=wp$_{\forall}$-skip]{%
  % i \notin \supp(\m{t}) % implied by well-formedness
}{
  \WPU {(\m{t} \m. \m[i: \code{skip}])} {Q}
  \lequiv
  \WPU {\m{t}} {Q}
}

    %!TEX root = ../main.tex
\infer*[lab=wp$_{\forall}$-prim$_{\oplus}$]{}{
  \V |- \WPU {\m[i: v_1 \oplus v_2]}
  {\ret. \ret[i] = (v_1 \mathbin{\sem{\oplus}} v_2)}
}

    %!TEX root = ../main.tex
\infer*[lab=wp$_{\forall}$-seq$_I$]{}{
  \WPU {\m[i: t_i | i \in I]}[\big]{
    \WPU {\m*[i: \smash{t'_i} | i\in I]} {Q}
  }
  \lequiv
  \WPU {\m[i: (t_i\code{;}\ t'_i) | i \in I]} {Q}
}

    %!TEX root = ../main.tex
\infer*[lab=wp$_{\forall}$-assign$_I$]{
  \forall i\in I\st
  (\p{x}_i,i) \not\in \pvar(Q)
}{
  \V
  \WPU {\m[i: e_i | i\in I]} {Q}
  |-
  \WPU {\m[i: \p{x}_i \code{:=}\, e_i | i \in I]}
  {\ret.Q(\ret) \land \LAnd_{i\in I} \ret[i] = \p{x}_i(i)}
}

    %!TEX root = ../main.tex
\infer*[lab=wp$_{\forall}$-if$_I$]{}{
  \WPU {\m[i: g_i | i\in I]}*{
    \fun\m{b}.
    \WPU {
      \begin{pmatrix*}[l]
        {\m[i: t_i | i\in I, \m{b}(i) \neq 0]}
        {\m.}\\
        {\m[i: t'_i | i\in I, \m{b}(i) = 0]}
      \end{pmatrix*}
    }[\big]{Q}
  }
  \lequiv
  \WPU {\m[i: \code{if}\ g_i\ \code{then}\ t_i\ \code{else}\ t'_i | i \in I]}
  {Q}
}

    %!TEX root = ../main.tex
\infer*[lab=wp$_{\forall}$-while$_I$]{
  \V
  P
  |-
  \WPU {\m[i: g_i | i\in I]}[\big]{
    \fun\m{b}.
    (\m{b} =_I 0 \land R)
    \lor
    (\m{b} \ne_I 0 \land \WPU {\m[i: t_i | i \in I]} {P})
  }
}{
  \V
  P
  |-
  \WPU {\m[i: \code{while}\ g_i\ \code{do}\ t_i | i \in I]} {R}
}
  \end{proofrules}
  \caption{Lockstep rules for $\mathbf{wp}_{\forall}$ from LHC}
\end{mathfig}

\begin{mathfig}{\small}
  \begin{proofrules}
    %!TEX root = ../main.tex
\infer*[lab=wp$_{\forall}$-nest]{%
  % \supp(\m{t}_1) \inters \supp(\m{t}_2) = \emptyset
  % implied by well definedness
}{
  % \WPU{\m{t}_1}{\WPU{\m{t}_2}{Q}} % simpler but discards ret vals of t1
  \WPU{\m{t}_1}{\fun\m{v}.
    \WPU{\m{t}_2}{\fun\m{w}.
      Q(\m{v}\m.\m{w})}}
  \lequiv
  \WPU{(\m{t}_1 \m. \m{t}_2)}{Q}
}

    %!TEX root = ../main.tex
\infer*[lab=wp$_{\forall}$-conj]{
  % \idx(Q_1) \subs \supp(\m{t}_1)
  \idx(Q_1) \inters \supp(\m{t}_2) \subs \supp(\m{t}_1)
  \\
  % \idx(Q_2) \subs \supp(\m{t}_2)
  \idx(Q_2) \inters \supp(\m{t}_1) \subs \supp(\m{t}_2)
}{
  \V
  \WPU{\m{t}_1}{Q_1}
  \land
  \WPU{\m{t}_2}{Q_2}
  |-
  \WPU{(\m{t}_1 \m+ \m{t}_2)}{Q_1 \land Q_2}
}


    %!TEX root = ../main.tex
\infer*[lab=wp$_{\forall}$-proj,
  Right={$ I = \supp(\m{t}_1) $}
]{
  %I = \supp(\m{t}_1)
}{
  \V
  \P I. \bigl(
  \proj(\m{t}_2) \implies \proj(\m{t}_1)
  \land
  \WPU{(\m{t}_1 \m. \m{t}_2)}{Q}
  \bigr)
  |-
  \WPU{\m{t}_2}{\PP I.Q}
}
% derivable from wp$\forall$-elim, proj-irrel, proj-weak, proj-intro
  \end{proofrules}
  \caption{Hyper-structure laws from LHC}
\end{mathfig}

\begin{mathfig}{\small}
  \begin{proofrules}
    %!TEX root = ../main.tex
\infer*[lab=wp$_{\forall}$-idx-pass]{
  i,j \notin \supp(\m{t})
}{
  \V (\WPU{\m{t}}{Q})\isub{j->i} |- \WPU{\m{t}}{Q\isub{j->i}}
}


    %!TEX root = ../main.tex
\infer*[lab=wp$_{\forall}$-idx-swap]{
  i \notin \idx(Q)
  % \\
  % i \notin \supp(\m{t}') % implied by well definedness
}{
  \V
  \bigl(\WPU{(\m[j: t]\m.\m{t}')}{Q}\bigr)\isub{j->i}
  |-
  \WPU{(\m[i: t]\m.\m{t}')}{Q\isub{j->i}}
}


    %!TEX root = ../main.tex
\infer*[lab=wp$_{\forall}$-idx-merge]{}{
  \V
  \bigl(\WPU{(\m[i: t, j: t]\m.\m{t}')}{Q}\bigr)\isub{j->i}
  |-
  \WPU{(\m[i: t]\m.\m{t}')}{Q\isub{j->i}}
}


    %!TEX root = ../main.tex
\infer*[lab=wp$_{\forall}$-idx-post]{
  \V \Gamma |- \WPU{\m{t}}{Q}
  \\
  j \notin \supp(\m{t}) \union \idx(\Gamma)
}{
  \V \Gamma |- \WPU{\m{t}}{Q\isub{j->i}}
}

  \end{proofrules}
  \caption{Reindexing rules from LHC}
\end{mathfig}