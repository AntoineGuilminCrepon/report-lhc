%!TEX root = main.tex
\section{Overview of LHC}
\label{sec:lhc}

This section is a summary of \citet{DOsualdo22}. We invite to refer to this paper for a more detailed overview.

\subsection{Preliminaries}

This section deals with notations, and defines hyperterms, and hyperassertions defining properties on these hyperprograms.

For the remainder of this paper, $\PVar$ is the set of variables, $\Val\is\Int$ the set of values and $\Idx\is\Nat$ the set of indexes for hyperobjects.

\begin{definition}[Finite maps]
For all sets $A, B$, we note $f: A \pto B$ a partial function from $A$ to $B$, \ie a function $f: A\rightarrow B_{\bot}$, where $B_{\bot} = B\cup\{\bot\}$. If $\supp(f) \is \{a \in A \mid f(a) \ne \bot\}$ is finite, we call it a \emph{finite map from $A$ to $B$}.
\end{definition}

We note $[a_1: b_1,\dots,a_n:b_n]$ the finite map associating each $a_i$ to the corresponding $b_i$. We define on the set of finite maps $A \pto B$ an opertion of union (noted $+$) in the following way.

\[
  (f + g)(x) = 
    \begin{cases}
      f(x) \CASE x \in \supp(f) \setminus g(x) \\
      g(x) \CASE x \in \supp(g) \\
      \bot \OTHERWISE \\
    \end{cases}
\]

The union is considered undefined if there exists $x \in \supp(f)\cap\supp(g)$ such that $f(x) \ne g(x)$. If $\supp(f)\cap\supp(g) = \emptyset$, we denote the union by $f \cdot g$. We define $f\m[a:b] \is \fun x. \code{if}\ x=a\ \code{then}\ b\ \code{else}\ f(x)$.

Furthermore, we define a \emph{reindexing} as any (potentially non-bijective) function of type $\Idx\rightarrow\Idx$. For any reindexing $\pi$ and any $f\of\Idx\rightarrow A$, one notes $f\isub*{\pi} = \fun i. f(\pi(i))$ the reindexing of $f$ through $\pi$.

We can now define the following types (with $\Store$ the type of stores, and $\Term$ the type of Terms).

\begin{definition}[Hyperobjects]
\begin{grammar}
  t \in\Term \is x \mid v \mid \code{rand} \mid e_1 \oplus e_2 \mid \code{if}\ g\ \code{then}\ t1\ \code{else}\ t_2 \mid \code{while}\ g\ \code{do}\ t \\
\end{grammar}
\begin{eqnarray*}
  \Store&\is&\PVar\rightarrow\Val \\
  \Type{HStore}&\is&\Idx\pto\Store \\
  \Type{HTerm}&\is&\Idx\pto\Term \\
  \Type{HReturn}&\is&\Idx\pto\Val \\
  \Assrt&\is&\Store\rightarrow\Prop \\
  \HAssrt&\is&\Type{HStore}\rightarrow\Prop \\
  \Type{PostHAssrt}&\is&\Type{HReturn}\rightarrow\HAssrt \\
\end{eqnarray*}
\end{definition}

Each term evaluates to a value and can mutate a store. We note $g$ the terms evaluating to booleans (\ie guards), and $e$ the terms evaluating to integers (\ie expressions). \code{rand} evaluates non-deterministically to an integer. Commands evaluates to an arbitrary (and irrelevant) value.

\begin{definition}[Semantics]
  For $t\in\Term$, $s,s'\in\Store$, $v\in\Val$, we define the judgment $\bigstep{t}{s}{v}{s'}$, indicating that an execution of $t$ with input store $s$ terminates and evaluates to $v$ with ouput store $s'$. The judgment is evaluated using a common bigstep semantics.

  This judgment can be extended to the hyperobjects; for $\m{t}\in\Type{HTerm}$, $\m{s},\m{s'}\in\Type{HStore}$, $\m{v}\in\Type{HReturn}$:
  \[
    \bigstep{\m{t}}{\m{s}}{\m{v}}{\m{s'}} \is \A i\in\Idx.
    \begin{cases}
      \bigstep{\m{t}(i)}{\m{s}(i)}{\m{v}(i)}{\m{s'}(i)} \CASE i\in\supp(\m{t}) \\
      \bot \OTHERWISE
    \end{cases}
  \]

  We also note $\bigsome{t}{s} \is \E v,s'. \bigstep{t}{s}{v}{s'}$ the judgment that $t$ may terminate with input store $s$. Note that it is equivalent to termination if $t$ is deterministic. This judgment is naturally extended to hyperterms.
\end{definition}

For any term $t$, $\pvar(t)$ (resp. $\mods(t)$) refers to the set of variables used (resp. modified) by $t$. They are extended to hyperterms as follows : $\pvar(\m{t}) \is \{(x,i) \mid i \in\supp(\m{t}), x\in\pvar(\m{t}(i))\}$ (defined similarly for mods).

Hyperassertions (of type $\HAssrt$) and posthyperassertions (of type $\Type{PHAssrt}$) are used in the hypertriple, respectively as pre- and postassertions.

For every hyperassertion $P$ and posthyperassertion $Q$, we define $\idx$ the set of indices relevant to them and $\pvar$ the set of variables relevant to them:

\begin{definition}[\/$\idx$]
\label{def:assrt-idx}
  \begin{align*}
    \idx(P) &\is
      \Idx \setminus
        \set{ i \in \Idx |
                \forall \m{s},s'\st P(\m{s}) \iff P(\m{s}\m[i: s']) }
    \\
    \idx(Q) &\is
      \Idx \setminus
        \set{ i \in \Idx |
                \forall \ret,\m{s},s'\st
                  Q(\ret)(\m{s}) \iff Q(\ret\m[i: \bot])(\m{s}\m[i: s']) }
  \end{align*}
\end{definition}

\begin{definition}[\/$\pvar$]
    \begin{align*}
    \pvar(P) &\is
      ( \PVar \times \Idx )
      \setminus
      \set*{
        (\p{x}, i) |
          \forall \m{s}, v\st
            P(\m{s}) \iff P\bigl(\m{s}\m[i: \m{s}(i){\m[\p{x}: v]} ]\bigr)
      }
    \\
    \pvar(Q) &\is\textstyle
      \Union_{\ret \from \Type{HReturn}} \pvar(Q(\ret))
  \end{align*}
\end{definition}

\cref{fig:hyper-assertions} gives notations for hyperassertions that are used later in the paper. Most of them are straightforward extension of common properties operators, but two of them deserves some precision. If we have an property $A$, it can be made into an hyperproperty, by precising the set of indices $I$ where it should hold, by noting $\at{A}{I}$. And if we consider an hyperassertion $P$, then $\P I. P$ is the hyperassertion obtained by ignoring any precision given by $P$ on indices outside of $I$ (as an example, $\P \{1, 2\}. (x(\I 1) = y(\I 2) \land x(\I 2) = y(\I 3)) \iff (x(\I 1) = y(\I 2) \land \E v. x(\I Q2) = v)$)

\begin{mathfig}[\small]
  \begin{align*}
    (\p{x}(i) = v) &\is
      \fun \m{s}.
        \m{s}(i)(\p{x}) = v
      % && \text{Indexed program variable}
    &
    P_1 \implies P_2 &\is
      \fun \m{s}.
        P_1(\m{s}) \implies P_2(\m{s})
      % && \text{Implication}
    \\
    P_1 \land P_2 &\is
      \fun \m{s}. P_1(\m{s}) \land P_2(\m{s})
      % && \text{Conjunction}
    &
    P_1 \lor P_2 &\is
      \fun \m{s}. P_1(\m{s}) \lor P_2(\m{s})
      % && \text{Disjunction}
    \\
    \E x.P(x) &\is
      \fun \m{s}.\exists x\st P(x)(\m{s})
      % && \text{Existential quantification}
    &
    \A x.P(x) &\is
      \fun \m{s}.\forall x\st P(x)(\m{s})
      % && \text{Universal quantification}
    \\
    P\isub*{\pi} &\is
      \fun \m{s}.
        % P(\m[i: \m{s}(\pi(i)) | i \in \Idx])
        P(\m{s}\isub*{\pi})
    &
    Q\isub*{\pi} &\is
      \ret.Q(\ret\isub*{\pi})\isub*{\pi}
      % && \text{Index substitution}
    \\
    \P I.P &\is
      \fun \m{s}.
        \exists \pr{\m{s}}\st
          P(\m{s}\m[i: \m{s}'\!(i) | i \in I])
    &
    \PP I.Q &\is \ret.\E \m{v}.\P I.Q(\ret\m[i: \m{v}(i) | i\in I])
      % && \text{Projection}
    \\
    \at{A}{I} &\is
      \fun \m{s}. \LAnd_{i\in I} A(\m{s}(i))
      % && \text{Assertion lifting}
    &
    Q_1 \land Q_2 &\is \ret. Q_1(\ret) \land Q_2(\ret)
    % \\
    % \pure{\phi} &\is \fun \wtv.\phi
    % &
    % \T{P}{\m{t}}{Q} &\is \pure*{\J |- {P}{\m{t}}{Q}}
   \end{align*}
  \caption{Hyper-assertions}
  \label{fig:hyper-assertions}
\end{mathfig}

\subsection{The $\wpsymb_{\forall}$ predicate}

We now introduces a new modality to the logic, directly inspired from the original Hoare logic: the weakest precondition predicate. This modality intuitevily means that all terminating traces of $\m{t}$ give values $\m{v}$ and output states $\m{s'}$ such that $Q(\m{v})(\m{s'})$ holds. More formally:

\begin{definition}[Weakest precondition]
  $\WPU{\m{t}}{Q} \is \fun \m{s}. \A \m{s'} \m{v}. \bigstep {\m{t}}{\m{s}} {\m{v}} {\m{s'}} \implies Q(\m{v})$
\end{definition}

We can now define hypertriples $\T {P} {\m{t}} {Q}$ as $P \implies \WPU {\m{t}} Q$. LHC mainly revolves around the behaviour of this $\wpsymb$ predicate, and the rules defining this behavior. We can divide them into four categories.

\begin{mathfig}[\small]
  \begin{proofrules}
    %!TEX root = ../main.tex
\infer*[lab=wp$_{\forall}$-triv]{}{
  \V |- \WPU{\m{t}}{\True}
}
    \label{rule:wpU-triv}

    %!TEX root = ../main.tex
\infer*[lab=wp$_{\forall}$-cons]{
  \forall \m{v}\st\ %
  \V Q(\m{v}) |- Q'(\m{v})
}{
  \V \WPU{\m{t}}{Q} |- \WPU{\m{t}}{Q'}
}
    \label{rule:wpU-cons}

    %!TEX root = ../main.tex
\infer*[lab=wp$_{\forall}$-all]{}
{
  \A x.\WPU{\m{t}}{Q(x)} \lequiv \WPU{\m{t}}{\A x.Q(x)}
}
%   \forall x\st\ %
%   \V \Gamma |- \WPU{\m{t}}{Q(x)}
% }{
%   \V \Gamma |- \WPU{\m{t}}{\A x.Q(x)}
% }
    \label{rule:wpU-all}

    %!TEX root = ../main.tex
\infer*[lab=wp$_{\forall}$-frame]{
  \pvar(P) \inters \mods(\m{t}) = \emptyset
}{
  \V P \land \WPU{\m{t}}{Q} |- \WPU {\m{t}} {P \land Q}
}
    \label{rule:wpU-frame}

    %!TEX root = ../main.tex
\infer*[lab=wp$_{\forall}$-impl-r]{
  \pvar(P) \inters \mods(\m{t}) = \emptyset
}{
  P \implies \WPU{\m{t}}{Q}
  \lequiv
  \WPU{\m{t}}{\ret. P \implies Q(\ret)}
}
    \label{rule:wpU-impl-r}

    %!TEX root = ../main.tex
\infer*[lab=wp$_{\forall}$-subst]{
  \p{x} \notin \mods(t)
}{
  \V
  \p{x}(i) = v
  \land
  \WPU{\bigl(\map[\big]{i: t\subst{\p{x}->v}}\m.\m{t}'\bigr)}{Q}
  |-
  \WPU{(\m[i: t]\m.\m{t}')}{Q}
}

    \label{rule:wpU-subst}

    %!TEX root = ../main.tex
\infer*[lab=wp$_{\forall}$-idx]{
  \pi\ \mathrm{bijective}
}{
  \V (\WPU{\m{t}}{Q})\isub*{\pi} |- \WPU{\m{t}\isub*{\pi}}{Q\isub*{\pi}}
}
    \label{rule:wpU-idx}
  \end{proofrules}
  \caption{Structural rules for $\wpsymb_{\forall}$ from LHC}
  \label{fig:structure-wpU-rules}
\end{mathfig}

\Cref{fig:structure-wpU-rules} describes \emph{structural} rules, that are direct adaptations of regular Hoare logic. \Cref{rule:wpU-triv,rule:wpU-cons,rule:wpU-all,rule:wpU-frame,rule:wpU-impl-r} state that $\wpsymb_{\forall}$ commmutes with $\True$, $\vdash$, $\forall$, $\land$, and ($P \implies$). \Cref{rule:wpU-subst} allows substitution and \cref{rule:wpU-idx} allows bijective reindexing.

\begin{mathfig}{\small}
  \begin{proofrules}
    %!TEX root = ../main.tex
\infer*[lab=wp$_{\forall}$-var]{}{
  \V |- \WPU { \m[i: \p{x}] } { \ret. \ret(i) = \p{x}(i) }
}
    \label{rule:wpU-var}

    %!TEX root = ../main.tex
\infer*[lab=wp$_{\forall}$-val]{}{
  \V |- \WPU {\m[i: v]}{\ret.\ret[i] = v}
}
    \label{rule:wpU-val}

    %!TEX root = ../main.tex
\infer*[lab=wp$_{\forall}$-skip]{%
  % i \notin \supp(\m{t}) % implied by well-formedness
}{
  \WPU {(\m{t} \m. \m[i: \code{skip}])} {Q}
  \lequiv
  \WPU {\m{t}} {Q}
}
    \label{rule:wpU-skip}

    %!TEX root = ../main.tex
\infer*[lab=wp$_{\forall}$-prim$_{\oplus}$]{}{
  \V |- \WPU {\m[i: v_1 \oplus v_2]}
  {\ret. \ret[i] = (v_1 \mathbin{\sem{\oplus}} v_2)}
}
    \label{rule:wpU-primop}

    %!TEX root = ../main.tex
\infer*[lab=wp$_{\forall}$-seq$_I$]{}{
  \WPU {\m[i: t_i | i \in I]}[\big]{
    \WPU {\m*[i: \smash{t'_i} | i\in I]} {Q}
  }
  \lequiv
  \WPU {\m[i: (t_i\code{;}\ t'_i) | i \in I]} {Q}
}
    \label{rule:wpU-seq}

    %!TEX root = ../main.tex
\infer*[lab=wp$_{\forall}$-assign$_I$]{
  \forall i\in I\st
  (\p{x}_i,i) \not\in \pvar(Q)
}{
  \V
  \WPU {\m[i: e_i | i\in I]} {Q}
  |-
  \WPU {\m[i: \p{x}_i \code{:=}\, e_i | i \in I]}
  {\ret.Q(\ret) \land \LAnd_{i\in I} \ret[i] = \p{x}_i(i)}
}
    \label{rule:wpU-assign}

    %!TEX root = ../main.tex
\infer*[lab=wp$_{\forall}$-if$_I$]{}{
  \WPU {\m[i: g_i | i\in I]}*{
    \fun\m{b}.
    \WPU {
      \begin{pmatrix*}[l]
        {\m[i: t_i | i\in I, \m{b}(i) \neq 0]}
        {\m.}\\
        {\m[i: t'_i | i\in I, \m{b}(i) = 0]}
      \end{pmatrix*}
    }[\big]{Q}
  }
  \lequiv
  \WPU {\m[i: \code{if}\ g_i\ \code{then}\ t_i\ \code{else}\ t'_i | i \in I]}
  {Q}
}
    \label{rule:wpU-if}

    %!TEX root = ../main.tex
\infer*[lab=wp$_{\forall}$-while$_I$]{
  \V
  P
  |-
  \WPU {\m[i: g_i | i\in I]}[\big]{
    \fun\m{b}.
    (\m{b} =_I 0 \land R)
    \lor
    (\m{b} \ne_I 0 \land \WPU {\m[i: t_i | i \in I]} {P})
  }
}{
  \V
  P
  |-
  \WPU {\m[i: \code{while}\ g_i\ \code{do}\ t_i | i \in I]} {R}
}
    \label{rule:wpU-while}
  \end{proofrules}
  \caption{Lockstep rules for $\wpsymb_{\forall}$ from LHC}
  \label{fig:lockstep-wpU-rules}
\end{mathfig}

\Cref{fig:lockstep-wpU-rules} describes \emph{lockstep} rules, \ie rules allowing for lockstep proofs, as described before. They are similar to Hoare logic rules, and we can in fact recover these corresponding Hoare logic rules by taking the arity-1 case. The \cref{rule:wpU-while} is stated as a rule, but we will discuss a possible derivation of it later.

\begin{mathfig}{\small}
  \begin{proofrules}
    %!TEX root = ../main.tex
\infer*[lab=wp$_{\forall}$-nest]{%
  % \supp(\m{t}_1) \inters \supp(\m{t}_2) = \emptyset
  % implied by well definedness
}{
  % \WPU{\m{t}_1}{\WPU{\m{t}_2}{Q}} % simpler but discards ret vals of t1
  \WPU{\m{t}_1}{\fun\m{v}.
    \WPU{\m{t}_2}{\fun\m{w}.
      Q(\m{v}\m.\m{w})}}
  \lequiv
  \WPU{(\m{t}_1 \m. \m{t}_2)}{Q}
}
    \label{rule:wpU-nest}

    %!TEX root = ../main.tex
\infer*[lab=wp$_{\forall}$-conj]{
  % \idx(Q_1) \subs \supp(\m{t}_1)
  \idx(Q_1) \inters \supp(\m{t}_2) \subs \supp(\m{t}_1)
  \\
  % \idx(Q_2) \subs \supp(\m{t}_2)
  \idx(Q_2) \inters \supp(\m{t}_1) \subs \supp(\m{t}_2)
}{
  \V
  \WPU{\m{t}_1}{Q_1}
  \land
  \WPU{\m{t}_2}{Q_2}
  |-
  \WPU{(\m{t}_1 \m+ \m{t}_2)}{Q_1 \land Q_2}
}

    \label{rule:wpU-conj}

    %!TEX root = ../main.tex
\infer*[lab=wp$_{\forall}$-proj,
  Right={$ I = \supp(\m{t}_1) $}
]{
  %I = \supp(\m{t}_1)
}{
  \V
  \P I. \bigl(
  \proj(\m{t}_2) \implies \proj(\m{t}_1)
  \land
  \WPU{(\m{t}_1 \m. \m{t}_2)}{Q}
  \bigr)
  |-
  \WPU{\m{t}_2}{\PP I.Q}
}
% derivable from wp$\forall$-elim, proj-irrel, proj-weak, proj-intro
    \label{rule:wpU-proj}
  \end{proofrules}
  \caption{Hyper-structure rules from LHC}
  \label{fig:hyperstructure-rules}
\end{mathfig}

\Cref{fig:hyperstructure-rules} describes \emph{hyperstructural} rules, and is the center of LHC's novelty. \Cref{rule:wpU-nest} allows to change the arity of an hyper-triple, by nesting $\wpsymb_{\forall}$ inside each other. \Cref{rule:wp-conj} gives another of combining two hypertriples (of potentially different arity) into another of a greater arity. \Cref{rule:wp-proj} describes the behavior of $\wpsymb_{\forall}$ with the projection operator $\Pi_I$. This rule uses a new jusgement, $\proj(\m{t})$, to maintain its soundness. This predicate is defined as follows (for hyperterms, there is a similar definition for unary terms):
\[
  \proj(\m{t}) \is \fun \m{s}. \bigsome{\m{t}}{\m{s}}
\]
Intuitevily, $\proj(\m{t})$ states that $\m{t}$ has at least one terminating trace. We will come back to this predicate in \cref{sec:extension}.

\begin{mathfig}{\small}
  \begin{proofrules}
    %!TEX root = ../main.tex
\infer*[lab=wp$_{\forall}$-idx-pass]{
  i,j \notin \supp(\m{t})
}{
  \V (\WPU{\m{t}}{Q})\isub{j->i} |- \WPU{\m{t}}{Q\isub{j->i}}
}

    \label{rule:wpU-idx-pass}

    %!TEX root = ../main.tex
\infer*[lab=wp$_{\forall}$-idx-swap]{
  i \notin \idx(Q)
  % \\
  % i \notin \supp(\m{t}') % implied by well definedness
}{
  \V
  \bigl(\WPU{(\m[j: t]\m.\m{t}')}{Q}\bigr)\isub{j->i}
  |-
  \WPU{(\m[i: t]\m.\m{t}')}{Q\isub{j->i}}
}

    \label{rule:wpU-idx-swap}

    %!TEX root = ../main.tex
\infer*[lab=wp$_{\forall}$-idx-merge]{}{
  \V
  \bigl(\WPU{(\m[i: t, j: t]\m.\m{t}')}{Q}\bigr)\isub{j->i}
  |-
  \WPU{(\m[i: t]\m.\m{t}')}{Q\isub{j->i}}
}

    \label{rule:wpU-idx-merge}

    %!TEX root = ../main.tex
\infer*[lab=wp$_{\forall}$-idx-post]{
  \V \Gamma |- \WPU{\m{t}}{Q}
  \\
  j \notin \supp(\m{t}) \union \idx(\Gamma)
}{
  \V \Gamma |- \WPU{\m{t}}{Q\isub{j->i}}
}

    \label{rule:wpU-idx-post}
  \end{proofrules}
  \caption{Reindexing rules from LHC}
  \label{fig:reindexing-rules}
\end{mathfig}

\Cref{fig:reindexing-rules} describes \emph{reindexing} rules, \ie the behaviour of $\wpsymb_{\forall}$ with non-bijective reindexings. Those reindexings can be reduced to binary reindexings $\isub{j->i}$, with $i\ne j$. \Cref{rule:wpU-idx-pass} deals with the case where $i,j\notin\supp(\m{t})$: the reindexing has no effect, it can simply be transmitted to the post-condition. \Cref{rule:wpU-idx-swap} deals with the case where $j\in\supp(\m{t})$ but $i\notin\supp(\m{t})$: the reindexing simply change the index of the corresponding term. \Cref{rule:wpU-idx-merge} allows merging two terms, if they are identical. \Cref{rule:wpU-idx-post} allows to directly transmit a reindexing to the postcondition, if the context $\Gamma$ doesn't constrain $j$.

\Cref{sec:coq-code} gives an (incomplete) implementation of the original logic, written during the first two months of my internship. The remaining three months were spent building the extension that will occupy the remainder of this paper.