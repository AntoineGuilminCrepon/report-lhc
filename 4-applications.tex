%!TEX root = main.tex
\section{Applications of the extended model}
\label{sec:applications}

\subsection{Writing specifications inside the logic}

When formalizing specifications for a program, there might be multiple ways to do it, and choosing which formalization to use is not always self-evident. One of the strengths of LHC is being able to discuss the relationships between those implementations inside of the logic itself.

As an overarching example, consider idempotence. Semi-formally, one defines a term $t$ as idempotent if $t \simeq (t;t)$. This specification can be divided into two refinement statements: $t \semleq (t;t)$ and $(t;t) \semleq t$. Thanks to the new extension, one can now formalize such refinement statements inside the logic:

\begin{align}
    \V \pv{x}(\I 1) = \pv{x}(\I 2) &|- \WPU {\m[\I 1: t]} {\WPE {\m[\I 2: t;t]} {\pv{x}(\I 1) = \pv{x}(\I 2)}} 
    \tag{\textsc{IdemSeq}$_{12}$}
    \\
    \V \pv{x}(\I 1) = \pv{x}(\I 2), \pv{x}(\I 3) = \pv{v} &|- \WPU {\m[\I 1: t]} {\WPE {\m[\I 2: t]} {\WPE {\m[\I 3: t]} {\pv{x}(\I 2) = \pv{v} \implies \pv{x}(\I 1) = \pv{x}(\I 3)}}} 
    \tag{\textsc{Idem}$^3_{12}$}
    \\
    \V \pv{x}(\I 1) = \pv{x}(\I 2) &|- \WPU {\m[\I 1: t; t]} {\WPE {\m[\I 2: t]} {\pv{x}(\I 1) = \pv{x}(\I 2)}} 
    \tag{\textsc{IdemSeq}$_{21}$}
    \\
    \V \pv{x}(\I 1) = \pv{x}(\I 3), \pv{x}(\I 2) = \pv{v} &|- \WPU {\m[\I 1: t]} {\WPU {\m[\I 2: t]} {\WPE {\m[\I 3: t]} {\pv{x}(\I 1) = \pv{v} \implies \pv{x}(\I 2) = \pv{x}(\I 3)}}} 
    \tag{\textsc{Idem}$^3_{21}$}
\end{align}

In those formalizations, we suppose that $\pvar(t) = \pvar(t;t) \subseteq \pv{x}$.

Of these four formalizations, the top two implement $t \semleq (t;t)$ and the bottoom two $(t;t) \semleq t$. Moreover, we have a formal definition of refinement outside of the logic:

\[
    t_1 \sqsubseteq t_2 \is \A s\ v\ s'. \bigstep {t_1} s v {s'} \implies \bigstep {t_2} s v {s'}
\]

And we can prove the following implications:

\[
\begin{array}{ccccc}
    t \sqsubseteq (t;t) &\Leftarrow& \textsc{IdemSeq}_{12} &\Leftarrow& \textsc{Idem}^3_{12} \\
    (t;t) \sqsubseteq t &\Leftarrow& \textsc{IdemSeq}_{21} &\Leftarrow& \textsc{Idem}^3_{21}
\end{array}
\]

\begin{proof}
    The leftmost implications are derived from this more general result on refinements :

    \begin{eqnarray*}
        t_1 \sqsubseteq t_2 &\is& \A s\ v\ s'. \bigstep {t_1} s v {s'} \implies \bigstep {t_2} s v {s'} \\
        &\Leftarrow& \A s\ v_1\ s_1'. \bigstep {t_1} s {v_1} {s_1'} \implies (\E v_2\ s_2'. \bigstep {t_2} s {v_2} {s_2'} \land s_2' = s_1') \\
        &\Leftarrow& \V \pv{x}(\I 1) = \pv{x}(\I 2) |- \WPU {\m[\I 1: t_1]} {\WPE {\m[\I 2: t_2]} {\pv{x}(\I 1) = \pv{x}(\I 2)}} \\
    \end{eqnarray*}

    The rightmost implications can be derived this way (the derivation is for $\textsc{Idem}_{12}^3 \implies \textsc{IdemSed}_{12}$, the other derivation is similar):

    \begin{prooftree}
        \AxiomC{$\V \pv{x}(\I 1) = \pv{x}(\I 2), \pv{x}(\I 3) = \pv{v} |- 
                \WPU {\m[\I 1: t]} {
                \WPE {\m[\I 2: t]} {
                \WPE {\m[\I 3: t]} {
                \pv{x}(\I 2) = \pv{v} \implies \pv{x}(\I 1) = \pv{x}(\I 3)}}}
        $}
        \UnaryInfC{$\V \pv{x}(\I 1) = \pv{x}(\I 2) |-
                \WPU {\m[\I 1: t]} {
                \WPE {\m[\I 2: t]} {\pv{x}(\I 2) = \pv{v} \land
                \WPE {\m[\I 2: t]} {\pv{x}(\I 1) = \pv{x}(\I 2)}}}
        $}
        \UnaryInfC{$\V \pv{x}(\I 1) = \pv{x}(\I 2) |- 
                \WPU {\m[\I 1: t]} {
                \WPE {\m[\I 2: t;t]} {\pv{x}(\I 1) = \pv{x}(\I 2)}} 
        $}
    \end{prooftree}
\end{proof}

\subsection{Loop Hoisting example}

Loop hoisting is a common compiler optimization that consists in reducing loops that are redundant in their action, because their body is idempotent (in the sense discussed in the previous part). To check the soundness of this optimization, we have to prove that if $t$ is idempotent (i.e $t \simeq (t;t)$) then $t^*;t \simeq t$ (some would argue that $t;t^*$ is mmore appropriate, but we will show later that this term is in fact equivalent to $t^*;t$, and the latter makes the derivation less cumbersome).

The refinement $t \semleq t^*;t$ is straightforward, by the following derivation:

\begin{prooftree}
    \AxiomC{}
    \UnaryInfC{$
        \V \pv{x}(\I 1) = \pv{x}(\I 2) |- \WPE {\m[\I 2: t^0]} {\pv{x}(\I 1) = \pv{x}(\I 2)}
    $}
    \UnaryInfC{$
        \V \pv{x}(\I 1) = \pv{x}(\I 2) |- \WPE {\m[\I 2: t^*]} {\pv{x}(\I 1) = \pv{x}(\I 2)}
    $}
    \AxiomC{$
        \V \pv{x}(\I 1) = \pv{x}(\I 2) |- \WPU {\m[\I 1: t]} {\WPE {\m[\I 2: t]} {\pv{x}(\I 1) = \pv{x}(\I 2)}}
    $}
    \BinaryInfC{$
    \V \pv{x}(\I 1) = \pv{x}(\I 2) |- \WPU {\m[\I 1: t]} {\WPE {\m[\I 2: t^*;t]} {\pv{x}(\I 1) = \pv{x}(\I 2)}} 
    $}
\end{prooftree}

The refinement $t^*;t \semleq t$ is derived as follows:

\begin{prooftree}
    \AxiomC{$
        \V \pv{x}(\I 1) = \pv{x}(\I 2) |- \WPU {\m[\I 1: t^*]} {P}
    $}
    \AxiomC{$
        \V P |- \WPU {\m[\I 1: t]} {\WPE {\m[\I 2: t]} {\pv{x}(\I 1) = \pv{x}(\I 2)}}
    $}
    \BinaryInfC{$
    \V \pv{x}(\I 1) = \pv{x}(\I 2) |- \WPU {\m[\I 1: t^*;t]} {\WPE {\m[\I 2: t]} {\pv{x}(\I 1) = \pv{x}(\I 2)}} 
    $}
\end{prooftree}

If we set $P := \E \pv{v}. \pv{x}(\I 1) = \pv{v} \land \WPU {\m[\I 1: t]} {\WPE {\m[\I 2: t]} {\pv{x}(\I 1) = \pv{x}(\I 2)}}$, the right-hand side becomes a tautology, and the left-hand side is derived as follows:

\begin{prooftree}
    \AxiomC{$
        \V \pv{x}(\I 1) = \pv{x}(\I 2) |- \WPU {\m[\I 1: t; t]} {\WPE {\m[\I 2: t]} {\pv{x}(\I 1) = \pv{x}(\I 2)}}
    $}
    \UnaryInfC{$
    \V \pv{x}(\I 1) = \pv{x}(\I 2) |- \WPU {\m[\I 1: t]}*
    {\begin{pmatrix*}[l]
        \E \pv{v}. \pv{x}(\I 1) = \pv{v} \land \\
        \WPU {\m[\I 1: t]} {\WPE {\m[\I 2: t]} {\pv{x}(\I 1) = \pv{x}(\I 2)}}
    \end{pmatrix*}}
    $}
    \UnaryInfC{$
    \V \pv{x}(\I 1) = \pv{x}(\I 2) |- \WPU {\m[\I 1: t^*]}*
    {\begin{pmatrix*}[l]
        \E \pv{v}. \pv{x}(\I 1) = \pv{v} \land \\
        \WPU {\m[\I 1: t]} {\WPE {\m[\I 2: t]} {\pv{x}(\I 1) = \pv{x}(\I 2)}}
    \end{pmatrix*}}
    $}
\end{prooftree}

It is of note that, for loop hoisting to be sound, one only needs to have the refinement $(t; t) \semleq t$.

As said before, the equivalence $t^*;t \simeq t;t^*$ is derivable. The first refinement $t;t^* \semleq t^*;t$ is derived this way:

\begin{prooftree}
    \AxiomC{$\V \pv{x}(\I 1) = \pv{x}(\I 2) |-
        \WPU {\m[\I 1: t]} {
        \overbrace{\WPE {\m[\I 2: t]} {\pv{x}(\I 1) = \pv{x}(\I 2)}}^P}
        $}
    \AxiomC{$\V \pv{x}(\I 1) = \pv{x}(\I 2) |-
    \WPU {\m[\I 1: t]} P$}
    \UnaryInfC{$\V P |- \WPE {\m[\I 1: t]} {\WPU {\m[\I 2: t]} P}$}
    \UnaryInfC{$\V P |- \WPU {\m[\I 1: t]} {\WPE {\m[\I 2: t]} P}$}
    \UnaryInfC{$\V P |- \WPU {\m[\I 1: t^*]} {\WPE {\m[\I 2: t^*]} P}$}
    \UnaryInfC{$\V \WPU {\m[\I 1: t]} P |- \WPU {\m[\I 1: t]} Q$}
    \BinaryInfC{$\V \pv{x}(\I 1) = \pv{x}(\I 2) |-
        \WPU {\m[\I 1: t]} {
        \underbrace{
        \WPU {\m[\I 1:t^*]} {
        \WPE {\m[\I 2:t^*]} {
        \WPE {\m[\I 2:t]} {\pv{x}(\I 1) = \pv{x}(\I 2)}}}}_Q}
        $}
\end{prooftree}

And the refinement $t^*;t \semleq t;t^*$ is derived that way:

\begin{prooftree}
    \AxiomC{$\V \pv{x}(\I 1) = \pv{x}(\I 2) |- Q$}
    \AxiomC{$\V Q |- \WPU {\m[\I 1: t]} Q$}
    \BinaryInfC{$\V \pv{x}(\I 1) = \pv{x}(\I 2) |-
        \WPU {\m[\I 1: t^*]} {
        \underbrace{
        \WPU {\m[\I 1:t]} {
        \WPE {\m[\I 2:t]} {
        \WPE {\m[\I 2:t^*]} {\pv{x}(\I 1) = \pv{x}(\I 2)}}}}_Q}
        $}
\end{prooftree}

The left-hand side, after choosing 0 iterations for the existential star, boils down to $\V \pv{x}(\I 1) = \pv{x}(\I 2) |- \WPU {\m[\I 1: t]} {\WPE {\m[\I 2: t]} {\pv{x}(\I 1) = \pv{x}(\I 2)}}$. The right-hand side is derived this way:

\begin{prooftree}
    \AxiomC{$\V \pv{x}(\I 1) = \pv{x}(\I 2) |-
                \WPU {\m[\I 1: t]} {
                \WPE {\m[\I 2: t]} {
                \pv{x}(\I 1) = \pv{x}(\I 2)}}$}
    \UnaryInfC{$\V \WPU {\m[\I 1: t]} {
                \WPE {\m[\I 2: t]} {
                \pv{x}(\I 1) = \pv{x}(\I 2)}}
            |- \WPU {\m[\I 1: t]} {
                \WPE {\m[\I 2: t]} {
                \WPU {\m[\I 1: t]} {
                \WPE {\m[\I 2: t]} {
                \pv{x}(\I 1) = \pv{x}(\I 2)}}}}$}
    \UnaryInfC{$\V \WPU {\m[\I 1: t]} {
                \WPE {\m[\I 2: t]} {
                \pv{x}(\I 1) = \pv{x}(\I 2)}}
            |- \WPU {\m[\I 1: t]} {
                \WPU {\m[\I 1: t]} {
                \WPE {\m[\I 2: t]} {
                \WPE {\m[\I 2: t]} {
                \pv{x}(\I 1) = \pv{x}(\I 2)}}}}$}
    \UnaryInfC{$\V Q |- \WPU {\m[\I 1: t]} Q$}
\end{prooftree}

\subsection{Dealing with $\exists\forall$ hyperproperties}

\citet{Dardinier23} presents an application of its logic where it deals with an hyperproperty with a quantifier alternation $\exists\forall$: the minimum. More precisely, it proves that if $t_1$ has a minimum, and $t_2$ is monotonous, then $t_1; t_2$ has a minimum. The derivation in this paper relies on the determinism of $t_2$. We will prove that this condition can be relaxed, only requiring $t_1; t_2$ to be projectable.

First, we write down the specifications needed for this proof:

\begin{align}
    \WPE {\m[\I 1: t_1]} {\WPU {\m[\I 2: t_1]} {\code{x}(\I 1) \leq \code{x}(\I 2)}}
    \tag{$hasMin_{\code{x}}(t_1)$} \\
    \code{x}(\I 1) \leq \code{x}(\I 2) \implies \WPU {\m[\I 1: t_2, \I 2: t_2]} {\code{y}(\I 1) \leq \code{y}(\I 2)}
    \tag{$Mono_{\code{x}\rightarrow \code{y}}(t_2)$}
\end{align}

We want to derive from those specifications (and $\proj(t_1; t_2)$) the goal:
\[
    \WPE {\m[\I 1: t_1; t_2]} {\WPU {\m[\I 2: t_1; t_2]} {\code{y}(\I 1) \leq \code{y}(\I 2)}}
\]

The derivation is as follows:

\begin{scprooftree}{0.9}
    \AxiomC{$\V |- \WPE {\m[\I 1: t_1]}
            {\WPE {\m[\I 1: t_2]} 
            {\True}}$}
    \AxiomC{$\V |- \WPE {\m[\I 1: t_1]}
                {\WPU {\m[\I 1: t_2]} 
                {\WPU {\m[\I 2: t_1]}
                {\WPU {\m[\I 2: t_2]} 
                {\code{y}(\I 1) \leq \code{y}(\I 2)}}}}$}
    \BinaryInfC{$\WPE {\m[\I 1: t_1]}
                {\WPE {\m[\I 1: t_2]} 
                {\WPU {\m[\I 2: t_1]}
                {\WPU {\m[\I 2: t_2]} 
                {\code{y}(\I 1) \leq \code{y}(\I 2)}}}}$}
    \UnaryInfC{$\V |- \WPE {\m[\I 1: t_1; t_2]} {\WPU {\m[\I 2: t_1; t_2]} {\code{y}(\I 1) \leq \code{y}(\I 2)}}$}
\end{scprooftree}

The left-hand side is simply $\proj(t_1; t_2)$. The right-hand side is derived the following way:

\begin{scprooftree}{0.9}
    \AxiomC{$\V |- \WPE {\m[\I 1: t_1]}
            {\WPU {\m[\I 2: t_1]} 
            {\code{x}(\I 1) \leq \code{x}(\I 2)}}$}
    \AxiomC{$\V \code{x}(\I 1) \leq \code{x}(\I 2) |-
            \WPU {\m[\I 1: t_2]}
            {\WPU {\m[\I 2: t_2]} 
            {\code{y}(\I 1) \leq \code{y}(\I 2)}}$}
    \BinaryInfC{$\V |- \WPE {\m[\I 1: t_1]}
                {\WPU {\m[\I 2: t_1]} 
                {\WPU {\m[\I 1: t_2]}
                {\WPU {\m[\I 2: t_2]} 
                {\code{y}(\I 1) \leq \code{y}(\I 2)}}}}$}
    \UnaryInfC{$\V |- \WPE {\m[\I 1: t_1]}
                {\WPU {\m[\I 1: t_2]} 
                {\WPU {\m[\I 2: t_1]}
                {\WPU {\m[\I 2: t_2]} 
                {\code{y}(\I 1) \leq \code{y}(\I 2)}}}}$}
\end{scprooftree}

To be completely frank, $\exists\forall$ hyperproperties are one of our logic's weaknesses, mainly because we don't have as nice lockstep rules as with $\forall\exists$ hyperproperties. Still, we hope that with this derivation, one can appreciate that LHC still has the expressiveness and the flexibility allowing to go around this lack of lockstep rules to solve certain goals of this form.